\section{Discussion}

Because of its design, we believe our platform is virtually capable of achieving the maximum scalability and availability that is feasible within the limitations of the third party services we use as components, with CrowdFlower likely being the first potential bottleneck. We could not find documentation of CrowdFlower deployments relying on the involvement of large volume of users\footnote{Crowdflower documents some of its success stories at \url{http://www.crowdflower.com/success-stories}.} [DID ELENA SAY SHE HAD VISIBILITY OF A CROWDFLOWER USE CASE THAT GAVE INDICATION OF THEIR CURRENT CAPACITY?]. We expect though that the size of the full OLAF problem would be manageable by the platform, particularly considering that we would be still using it more as a complement to ingestion of primary data sources and computation rather than as the main channel to generate data.

Conversely, the decision to rely only on their native functionality, forced us to work around several of CrowdFlower's design characteristics that were not compatible with our crowdsourcing model. This forced us to implement the workflow in a way that was far from ideal and spend much more than necessary on Worker fees, e.g. because of Workers could not be prevented to judge the same road more than once. The detailed discussion of these issue is outside of the scope of this document, but the break-even point between relying on a crowdsourcing platform native functionality vs the cost and complexity of full customisation makes an interesting subject for further research.

As showed by the results, a much more pressing problem is the design of the crowdsourcing component. The model we used for evaluation in this paper failed to catalyse the contributors' agreement around results that are statistically credible. For sure, the complexity of the task and the open-ended nature of the survey activity were the root cause. Gottlieb {\it et al.} in \cite{Gottlieb:2012fh} expressed similar concern when exploring crowdsourcing to geolocate video.

Finally, even if we were successful, any discussion should still be filtered through a cost / benefit examination: what is a "reasonable cost" of producing one address with a target degree of confidence? Has a solution based on anything but VGI any opportunity to be cost effective? We may find that out the complexity of producing and maintaining an address file, to paraphrase the UK law, actually makes PAF's licensing price "reasonable".
