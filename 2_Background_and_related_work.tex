\section{Background and related work}

\subsection{Crowdsourcing geospatial information}

Geographical Information Systems (GIS) are systems designed to capture, manipulate, analyse and visualise geographical data. 

Crowdsourcing geographical data was made possible in the early 2000's by the availability of GIS to the wider public. As web technology matured, publicly available, web-based GIS systems emerged including OpenStreetMap\footnote{Created in 2004, see \url{https://www.openstreetmap.org}.} (OSM), Google Maps\footnote{Launched in 2005, see \url{https://www.google.com/maps}. Google Earth is actually precedent to Google Maps and was launched in 2001, but at the time it was available as a downloadable desktop application only.} and Wikimapia\footnote{Created in 2006, see \url{http://wikimapia.org/}.}. In what was by some called "the democratisation of GIS" \cite{Butler:2006fe} laypeople could for the first time contribute to the systems, augmenting pre-existing data and creating new. This was around the same time the term "crowdsourcing" was used first.\footnote{J. Howe, "The Rise of Crowdsourcing" Wired, 01-Jun-2006.}

Crowdsourcing geospatial data became the subject of extensive research. Because of the origins of the phenomenon, volunteered rather than paid participation to the systems was studied in particular, to the point of defining the discipline itself. In 2007 Michael F. Goodchild coined the term "Volunteered Geographic Information" (Volunteered GI, or VGI) \cite{Goodchild:2007vt} and examined it as a new form of citizen science. Goodchild also was among the first to make the hypothesis that relying on the crowd could be more cost-effective at maintaining geospatial data than any of the traditional practices.

\subsection{Challenges of crowdsourcing geospatial data}
\label{challenges-of-crowdsourcing-geospatial-data}

There are a number of recognised challenges in crowdsourcing geospatial data, most of which are in common with crowdsourcing in general. The most relevant to our research are summarised here.

\textbf{Contributor credibility} The trustworthiness and expertise of a contributor defines her "credibility". In cases where crowdsourced GIS systems are designed to be accessible to the layperson, the only variable remains trustworthiness, which depends on the contributor's motivation, which in turn "suggests greater or less potential for bias and deception" \cite{Flanagin:2008ck}. Gatekeeping and quality control are practices used to assure contributor credibility. 

In our research we attempt at reduce bias by using paid crowdsourcing.

\textbf{Data reliability} Even assuming that the contributors are credible, it is necessary to assess the quality of the system's overall output. 

The reliability of mainstreams VGI systems was assessed for example by Haklay in \cite{Haklay:2010vs}, who performed a comparative study of data offered by OSM and the British NMCA Ordnance Survey (OS). In 2008, only a few years after it was started, OSM offered a "reasonable accuracy" of about 6 meters and an overlap of up to 100\% of roads in OS' data. 

Further research in \cite{Haklay:2010wf} suggests that the volume of volunteers involved in a project - even when working without a central coordination - can be considered as an intrinsic quality assurance measure.

In our research, paid crowdsourcing gave us more control over the volume of contributors we could engage, and enabled us to use robust statistical tools.

\textbf{Data completeness} Volunteered GI has a completeness issue, in the words of OSM creator Steve Coast: "Nobody wants to do council estates".\footnote{"The GISPro Interview with OSM founder Steve Coast" GIS Professional, no. 18, 2007.} Loose organisation of the contributors and a range of socioeconomic barriers \cite{Haklay:2010vs} may hinder achievement of sufficient coverage of those geographical areas volunteers are not motivated to contribute about.

Moreover, some applications may require the data to be produced within a given timeframe, not compatible with volunteer best effort, something that could be better controlled using paid crowdsourcing.

\textbf{Domain knowledge} Contributors to GI crowdsourcing projects need a sufficient understanding of the concepts that define the domain, such as what a road is, what a building etc. \cite{Ballatore:2015kg}. 

In our work we tried minimising this requirement, by simplifying the contributor task down to making simple observations from imagery.

\subsection{Open data and open GI}
\label{open-data-and-gi}

Because research has focussed on systems that leverage volunteer contribution and are designed for the pubic good, GI is often implicitly associated to open data: "data that anyone can access, use and share"\footnote{See \url{http://theodi.org/faq}.}. 

Government-funded NMCAs are natural owners of geospatial data and are commonly expected to release it in the open for the public good. In Great Britain, for example, since 2015 OS has opened a substantial volume of data that was previously available to the public as commercial products only, including datasets such as "Open Names" - a place-name index - and "Open Roads" - the generalised geometry and network connectivity of the road network - which we used in our platform as primary data sources.
	
The availability of such high quality and authoritative sources becomes a substantial enabler for the creation of new data and augmenting the existing.

All useful and needed geospatial data is not always made available in the open, though. Demand for open data can be suppressed, for example, due to failure in recognising open data-enabled business models, restrictive legislative and patent systems, or charging for access.\footnote{N. Shadbolt, "A Cornerstone for Open Data: The Postcode Address File" Apr-2013. Online. Available: \url{http://theodi.org/blog/cornerstone-open-data-postcode-address-file}. Accessed: 09-May-2015.} The OLAF problem, described below, is an example of this.
	
\subsection{The Open Legal Address File (OLAF)}
\label{subs:the-problem-of-creating-an-olaf}

The main use case for our research is creating an Open Legal Address File\footnote{See \url{https://github.com/Digital-Contraptions-Imaginarium/OLAF-yr2_lab/blob/gh-pages/docs/README.md#legal-address-files} for an explanation of the "legal address" term.}: a dataset that lists all known valid addresses and postcodes for the UK and is functionally equivalent to the "Postcode Address File" (PAF\footnote{PAF is a registered trademark by Royal Mail plc. For convenience we won't show the registered trademark sign "\textregistered" in this document every time we refer to it.}).

PAF is the ownership of Royal Mail, the UK ex state-owned postal service. Law makes PAF available on "reasonable terms" to "any person who wishes to use it".\footnote{See the Postal Service Act 2000, part VII, article 116.} This, however, never translated into making the data open. PAF is available as a commercial product only, and was sold by the government as part of Royal Mail's privatisation in October 2013. OLAF aims to fill this gap in the UK open geospatial data offering. 

Without going into the detail of official British Standard BS7666,\footnote{See \url{http://shop.bsigroup.com/ProductDetail/?pid=000000000030127201}.} a practical operational definition of a "valid UK address" would be {\it the set of unambiguous information needed to instruct any delivery service operator to deliver a piece of mail at a given public address (a private house, a commercial establishment etc.), from a consumer perspective}.\footnote{Find the reference address data model at \url{https://github.com/Digital-Contraptions-Imaginarium/OLAF-yr2_lab/tree/gh-pages/docs#address-format}.} 
