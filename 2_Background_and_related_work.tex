\section{Background and related work}

\subsection{Crowdsourcing geospatial information}

Geographical Information Systems (GIS) are systems designed to capture, manipulate, analyse and visualise geographical data. 

Crowdsourcing geographical data was made possible in the early 2000's by the availability of GIS to the wider public. As web technology matured, publicly available, web-based GIS systems emerged including OpenStreetMap\footnote{Created in 2004, see \url{https://www.openstreetmap.org}.} (OSM), Google Maps\footnote{Launched in 2005, see \url{https://www.google.com/maps}. Google Earth is actually precedent to Google Maps and was launched in 2001, but at the time it was available as a downloadable desktop application only.} and Wikimapia\footnote{Created in 2006, see \url{http://wikimapia.org/}.}. In what was by some called "the democratisation of GIS" \cite{Butler:2006fe} laypeople could for the first time contribute to the systems, augmenting pre-existing data and creating new. This was around the same time the term "crowdsourcing" was used first\footnote{J. Howe, "The Rise of Crowdsourcing" Wired, 01-Jun-2006.}.

Crowdsourcing geospatial data became the subject of extensive research. Because of the origins of the phenomenon, volunteered rather than paid participation to the systems was studied in particular, to the point of defining the discipline itself. In 2007 Michael F. Goodchild coined the term "Volunteered Geographic Information" (Volunteered GI, or VGI) \cite{Goodchild:2007vt} and examined it as a new form of citizen science. Goodchild also was among the first to make the hypothesis that relying on the crowd could be more cost-effective at maintaining geospatial data than any of the traditional ways.

\subsection{Challenges of crowdsourcing geospatial data}
\label{challenges-of-crowdsourcing-geospatial-data}

There are a number of recognised challenges in crowdsourcing geospatial data. The most relevant to our research are summarised here.

\textbf{Contributor credibility} The trustworthiness and expertise of a contributor defines her "credibility". In cases where crowdsourced GIS systems are designed to be accessible to the layperson and do not require specific expertise, the only credibility variable remains trustworthiness, which depends on the contributor's motivation, which in turn "suggests greater or less potential for bias and deception" \cite{Flanagin:2008ck}. Gatekeeping and quality control are practices used to assure contributor credibility. 

In our research paid crowdsourcing was used to cancel out bias.

\textbf{Data reliability} Even assuming that the contributors are individually credible, it is necessary to assess the quality of the system's overall output. 

The reliability of mainstreams VGI systems was assessed for example by Haklay in \cite{Haklay:2010vs}, who performed a comparative study of data offered by OSM and the British NMCA Ordnance Survey (OS). In 2008, only a few years after OSM was started, OSM offered a "reasonable accuracy" of about 6 meters and an overlap of up to 100\% of roads in OS' data. 

Further research in \cite{Haklay:2010wf} suggests that the volume of volunteers involved in a project - even when working without a central coordination - can be considered as an intrinsic quality assurance measure.

In our research, using paid crowdsourcing gave us more direct control over the volume of contributors we could engage on creating or validating the data and enabled us to use more robust statistical tools.

\textbf{Data completeness} Volunteered GI has a completeness issue, in the words of OSM creator Steve Coast: "Nobody wants to do council estates"\footnote{"The GISPro Interview with OSM founder Steve Coast" GIS Professional, no. 18, 2007.}. Loose organisation of the contributors and a range of socioeconomic barriers \cite{Haklay:2010vs} may hinder achievement of sufficient coverage of those geographical areas volunteers are not motivated to contribute about.

Moreover, some applications may require the data to be produced within a given timeframe, not compatible with volunteer best effort, something that could be better controlled using paid crowdsourcing.

\textbf{Domain conceptualisation} Contributors to GI crowdsourcing projects need a sufficient understanding of the concepts that define the domain, such as what a road is, what a building etc. \cite{Ballatore:2015kg}. 

In our work we tried minimising this requirement, by simplifying the contributor task down to making simple observations from imagery of the surveyed locations.

[PERHAPS SOMETHING ON GI AND HIBRYD WORKFLOWS CAN COME FROM \cite{Rice:2012vy} ]

\subsection{Open data and open Geographic Information}
\label{open-data-and-gi}

Because research has focussed on systems that leverage volunteer contribution and are designed for the pubic good, GI is often implicitly associated to open data: "data that anyone can access, use and share"\footnote{\url{http://theodi.org/faq}.}. 

Government-funded NMCAs are natural owners of geospatial data and are commonly expected to release it in the open for the public good. In Great Britain, for example, since 2015 the local NMCA Ordnance Survey (OS) has opened a substantial volume of data that was previously available to the public as commercial products only, e.g. products "Open Names", a place-name index, and "Open Roads", the generalised geometry and network connectivity of the road network.
	
The availability of such high quality and authoritative sources becomes a substantial enabler for the creation of new, original geospatial data and augmenting what is already available.

All useful and needed geospatial data is not always made available in the open, though. Demand for open data can be suppressed, for example, due to failure in recognising open data-enabled business models, restrictive legislative and patent systems, or charging for access\footnote{N. Shadbolt, "A Cornerstone for Open Data: The Postcode Address File" Apr-2013. Online. Available: \url{http://theodi.org/blog/cornerstone-open-data-postcode-address-file}. Accessed: 09-May-2015.}. The OLAF problem, described later in this document, is an example of this.
	
One of the assumptions at the base of our research is that the people's contribution is useful and necessary to work around this obstacles. People can be enabled, through technology, to capture and curate data, alongside what is already published by governments and businesses. Human-machine hybrid systems such as the one proposed in this paper can be built to facilitate this process.

\subsection{The Open Legal Address File (OLAF)}
\label{subs:the-problem-of-creating-an-olaf}

\textbf{The Postcode Address File} The main use case for our research is creating an Open Legal Address File\footnote{The term "legal addresses" in OLAF refers to all addresses that are, by law, in the public domain, hence have no restrictions in terms of intellectual property or privacy protection and can be published as open data. Legal addresses belong to either or both of the following two categories: a) they are the addresses of current or past UK residents who are or were registered on the public electoral register, and b) they are the addresses of past or present UK companies that are or were registered at the relevant British registrar, such as Companies House for England and Wales.} for the UK: a dataset that is functionally equivalent, for example, to Royal Mail's "Postcode Address File" ("PAF"\footnote{PAF is a registered trademark by Royal Mail plc. For convenience we won't show the registered trademark sign "\textregistered" in this document every time we refer to it.}). PAF is a commercial dataset that lists all known valid addresses and postcodes for the UK and was created by the once state-owned mail delivery services company over many years of operations.

An act of law\footnote{The Postal Service Act 2000, part VII, article 116, available at \url{http://www.legislation.gov.uk/ukpga/2000/26/contents}.} makes PAF ownership of Royal Mail and requires them to make PAF available on "reasonable terms" to "any person who wishes to use it". To this day, though, the "reasonable terms" did not translate into making PAF open data. The dataset is available from Royal Mail and its resellers only as a commercial product. This is considered an anomaly by many, as "reasonable terms for the external use of PAF data by third parties should be no more than the marginal cost of distribution (...)"\footnote{Open Data User Group, "Response to RM Postcode Address File licensing consultation" Online, last accessed on 14/08/2014, \odugurl.}.

In October 2013 Royal Mail and its assets were privatised\footnote{Royal Mail is currently a public limited company with only a 30\% of shares controlled by the Government. The Postal Services Act 2011, available at \url{http://www.legislation.gov.uk/ukpga/2011/5/contents}, plans to reduce the Government control down to 10\%.}, including PAF. The last reasonable expectation that the PAF could be made open data was lost. The UK Parliament House of Commons' Public Administration Select Committee called the sale of PAF "unacceptable and unnecessary" and recognised that PAF's "disposal for a short-term gain will impede economic innovation and growth"\footnote{UK House of Commons, Public Administration Select Commission, "Statistics and Open Data: Harvesting unused knowledge, empowering citizens and improving public services", Online, last accessed on 14/08/2014, \url{http://www.publications.parliament.uk/pa/cm201314/cmselect/cmpubadm/564/564.pdf}, Mar. 2014.}. Trying fill the gap left in the UK open geospatial data offering by this critical dataset makes an ideal case study. 

\textbf{The OLAF data model} It is useful to give a high level specification of the OLAF data model to clarify what an ideal dataset would look like. OLAF is a flat list of valid UK addresses. For the objectives of this study, and without going into the detail of official standards such as BS7666\footnote{See \url{http://shop.bsigroup.com/ProductDetail/?pid=000000000030127201}.}, it is possible to give a practical, operational definition of what a "valid UK address" is, that is, in addition to the name of the addressee, {\it the set of unambiguous information needed to instruct any delivery service operator to deliver a piece of mail at a given public address (a private house, a commercial establishment etc.), from a consumer perspective}. 

An address is characterised by the following properties\footnote{These are actually a subset of BS7666, that includes elements consumers are not required to specify in an address such as the UPRN: a nationally unique number assigned by the UK National Land \& Property Gazetteer to local authorities in order to give a unique identity to any addressable piece of land or property. UPRNs are also not available as open data.}:

\begin{itemize}
    \item A "primary addressable object name" (PAON), e.g. "10". The PAON identifies one property.
    \item A "secondary addressable object name" (SAON), e.g. "Flat 2". The SAON identifies a part of a property and is optional.
    \item A street name\footnote{For simplicity, note that the terms "place", "road" or "street" are used interchangeably here and in the rest of this document, as they are equivalent from an OLAF perspective. The "street name" in the model above, for example, may be the name of a square, etc.}, e.g. "Downing Street".
    \item A town name, e.g. "London".
    \item An UK postcode, e.g. "SW1A 2AA".
\end{itemize}

The elements of an address are not case sensitive and tolerant of abbreviations and alternative spellings and punctuation used in common British English language (e.g "Downing Street" is the same as "DOWNING ST.").
