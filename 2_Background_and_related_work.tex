\section{Background and related work}

\subsection{Crowdsourcing geospatial information}

Geographical Information Systems (GIS) are extensively studied in literature. They are systems designed to capture, manipulate, analyse and visualise geographical data, that traditionally is produced and maintained by professional organisations or institutions such as the NMCAs. 

In the early 2000's, the consolidation of web technologies permitted the emergence of publicly available, web-based GIS systems such as OpenStreetMap\footnote{Created in 2004, see \url{https://www.openstreetmap.org}.} ("OSM" in the following), Google Maps\footnote{Launched in 2005, see \url{https://www.google.com/maps}. Google Earth is actually precedent to Google Maps and was launched in 2001, but at the time it was available as a downloadable desktop application only, see \url{https://www.google.co.uk/intl/en_uk/earth/explore/products/desktop.html}.} and Wikimapia\footnote{Created in 2006, see \url{http://wikimapia.org/}.}. At the same time, non-GIS applications started integrating geospatial data into their features, too; e.g. photo publishing platform Flickr\footnote{See \url{https://www.flickr.com/}.} permitted its users to upload and locate their media on interactive maps. 

Not only these made GIS technology available to the larger public, but allowed non professionals to contribute to the systems, either augmenting pre-existing data or creating new data, in a transformation that some called "the democratisation of GIS" \cite{Butler:2006fe}. This was around the same time the term "crowdsourcing" itself was first used\footnote{See \url{http://www.wired.com/2006/06/crowds/}.}.

Crowdsourcing geospatial data started being the subject of extensive study. In 2007 Michael F. Goodchild examined the phenomenon as a new form of citizen science and coined the term "Volunteered Geographic Information" (Volunteered GI, or VGI) in his "Citizens as sensors: the world of volunteered geography" article \cite{Goodchild:2007vt}. In the same paper, Goodchild started examining the contributors' motivation and also proposed the hypothesis that relying on the crowd could be much more cost-effective at maintaining geospatial data than any of the traditional ways. 

 \subsection{Challenges of crowdsourcing geospatial data}

[STILL WRITING HERE, EXPANDING THE TEXT AS I RECOLLECT RELEVANT ITEMS FROM THE LITERATURE REVIEW]

\textbf{Contributor credibility} Both VGI and paid crowdsourcing share the problem of the contributors' credibility. This was examined for example in \cite{Flanagin:2008ck}. Although Flanagin and Metzger's work is focused on VGI only, crowdsourcing GI in general shares the same problems of data gatekeeping, quality control and credibility in general of the contributors and their output. 

Harvey \cite{Harvey:2012wsa} also highlights a distinction between VGI vs GI that is collected by systems as a side-effect of the use of some other service, typically mobile network use or GPS-enabled applications, that he calls "contributed" GI. Harvey argues that understanding the nature of the sources of crowdsourced data is instrumental to identify possible biases. As an example, he offers a scenario where a local activist group draws on OSM data to show the location of squats, only to find out that the data has been volunteered by the police force. 

\textbf{Data reliability} Even assuming that the contributors are sufficiently credible, the quality of their work may anyway not be sufficiently reliable for the target application. Haklay in \cite{Haklay:2010vs} performed a comparative study of the data offered by OSM and the British NMCA Ordnance Survey ("OS" in the following). Already in 2008, he found that OSM offered a reasonable accuracy of about 6m and an overlap of up to 100\% of roads in OS' data. 

Moreover, although the quality of the data was inconsistent and heavily dependent on the contributor rather than the complexity of the spatial unit, further research in  \cite{Haklay:2010wf} suggested that, the volume of contributors - even when working without a central coordination - can be considered as an intrinsic quality assurance measure, in a way analogous to what happens with open source software.

Crowdsourcing GI is seen a the cost effective and "good enough" solution to [SOMETHING] [REFERENCE]. 

\textbf{Data completeness} Volunteered GI in particular also has a completeness issue. Haklay attributes this to the loose organisation of the contributors \cite{Haklay:2010vs} [BLAH BLAH] and socioeconomic barriers: at the time when physical survey of the streets was still one of the main channels to create new data, OSM creator Steve Coast stated in an interview that "Nobody wants to do council estates" \cite{Anonymous:2007ux} and in general there will always be places volunteers won't be interested in contributing to.

Moreover, even if completeness could be achieved eventually, depending on the application the data may be useful only if it is timely available.

\textbf{Motivation and incentives} The volume of literature on VGI in respect to non-volunteered crowdsourcing of GI appears to represent some degree of mistrust in the possibility that the latter kind of system could be successful, or worth exploring in general in respect to traditional practices for geospatial data creation and maintenance.   

[WHERE TO PUT THIS? PAID CROWDSOURCING CAN BE FASTER THAN VGI AND RELYING ON A CROWD THAT IS DETACHED BY THE PLACES CAN ASSURE THE LACK OF BIAS]

\subsection{Open data and open Geographic Information}

GI is often associated to open data: data that anyone can access, use and share\footnote{\url{http://theodi.org/faq}.}. 

Government-funded NMCAs are natural owners of geospatial data and are commonly expected to release it in the open for the public good. In Great Britain, for example, since 2015 the local NMCA Ordnance Survey\footnote{See \url{http://www.ordnancesurvey.co.uk/}.} has opened a substantial volume of data that was previously available to the public as commercial products only, e.g. "Open Names", a place-name index, and "Open Roads", the generalised geometry and network connectivity of the road network.
	
The availability of such high quality and authoritative sources becomes a substantial enabler for the creation of new, original geospatial data. There where GI could only be created from scratch - as in OpenStreetMap's case when it was started in the UK in 2007 - it is now possible to rather complementing and augmenting what is already available.

All useful and needed geospatial data is not always made available in the open, though. Demand and offer for open data can be suppressed, for example, due to failure in recognising open data-enabled business models, restrictive legislative and patent systems, or charging for public datasets \cite{shadboltpaf}. The OLAF problem, described later in this document, is a great example of this.
	
One of the assumption at the base of our research is that the people's contribution is useful and necessary to work around this blockages. People can be enabled, through technology, to capture and curate data, alongside what is already published by governments and businesses. Human-machine hybrid systems such as the one proposed in this paper can be built to facilitate this process.

\subsection{The Open Legal Address File (OLAF)}
\label{subs:the-problem-of-creating-an-olaf}

\textbf{The Postcode Address File} The main use case for this research is creating an Open Legal Address File (OLAF) for the UK: a dataset that is functionally equivalent to the "Postcode Address File" or "PAF"\footnote{PAF is a registered trademark by Royal Mail plc. For convenience we won't show the registered trademark sign "\textregistered" in this document every time we refer to it.}: a commercial dataset that lists all known valid addresses and postcodes for the UK and was created over many years of operations by the once state-owned mail delivery services company Royal Mail.

An act of law\footnote{The Postal Service Act 2000, part VII, article 116 \cite{postalserviceact2000}.} makes PAF ownership of Royal Mail and requires them to make PAF available on "reasonable terms" to "any person who wishes to use it". To this day, though, the "reasonable terms" did not translate into making PAF open data. The dataset is available from Royal Mail and its resellers only as a commercial product. This is considered an anomaly by many, as "reasonable terms for the external use of PAF data by third parties should be no more than the marginal cost of distribution (...)" \cite{odugresponse}. 

In October 2013 Royal Mail and its assets were privatised\footnote{Royal Mail is currently a public limited company with only a 30\% of shares controlled by the Government. The Postal Services Act 2011 \cite{postalserviceact2011} plans to reduce the Government control down to 10\%.}, including PAF. The UK Parliament House of Commons' Public Administration Select Committee called the sale of PAF "unacceptable and unnecessary" and recognised that PAF's "disposal for a short-term gain will impede economic innovation and growth" \cite{pascod}.

The opportunity to create an alternative to PAF is an ideal case study. The term "legal addresses" in OLAF refers to all addresses that are, by law, in the public domain, hence have no restrictions in terms of intellectual property or privacy protection and can be published as open data\footnote{Legal addresses belong to either or both of the following two categories: a) they are the addresses of current or past UK residents who are or were registered on the public electoral register, and b) they are the addresses of past or present UK companies that are or were registered at the relevant British registrar, such as Companies House for England and Wales.}.

\textbf{Problem specifications} It is useful to give a high level specification of OLAF to clarify what the expected end result is. 

OLAF is a list of valid UK addresses. For the objectives of this study, and without going into the detail of official standards such as BS7666\footnote{See \url{http://shop.bsigroup.com/ProductDetail/?pid=000000000030127201}.}, it is possible to give a practical, operational definition of what a "valid UK address" is, that is, in addition to the name of the addressee, {\it the set of unambiguous information needed to instruct any delivery service operator to deliver a piece of mail at a given public address (a private house, a commercial establishment etc.), from a consumer perspective}. 

An address is characterised by the following properties\footnote{These are actually a subset of BS7666, that includes elements consumers are not required to specify in an address such as the UPRN: a nationally unique number assigned by the UK National Land \& Property Gazetteer to local authorities in order to give a unique identity to any addressable piece of land or property. UPRNs are also not available as open data.}:

\begin{itemize}
    \item A "primary addressable object name" (PAON), e.g. "10". The PAON identifies one property.
    \item A "secondary addressable object name" (SAON), e.g. "Flat 2". The SAON identifies a part of a property and is optional.
    \item A street name\footnote{For simplicity, note that the terms "place", "road" or "street" are used interchangeably here and in the rest of this document, as they are equivalent from an OLAF perspective. The "street name" in the model above, for example, may be the name of a square, etc.}, e.g. "Downing Street".
    \item A town name, e.g. "London".
    \item An UK postcode, e.g. "SW1A 2AA".
\end{itemize}

The elements of an address are not case sensitive and tolerant of abbreviations and alternative spellings and punctuation used in common British English language (e.g "Downing Street" is the same as "DOWNING ST.").

[ELENA: ADD A DESCRIPTION OF THE CHALLENGES OF THIS SPECIFIC PROBLEM]

