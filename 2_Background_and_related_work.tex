\section{Background and related work}

\subsection{Crowdsourcing geospatial information}

[TO WRITE]

Crowdsourcing geospatial data is extensively studied in in literature [BLAH BLAH AND REFERENCES], particularly when performed by volunteers, under the name of Volunteered Geographic Information (Volunteered GI, or VGI) [BLAH BLAH AND REFERENCES], also thanks to use cases such as Wikimapia\footnote{\url{http://wikimapia.org/}.} or OpenStreetMap\footnote{\url{http://www.openstreetmap.org/}.}. The latter is likely the best known VGI-based mapping service available today.

Crowdsourcing GI is seen a the cost effective and "good enough" solution to [SOMETHING] [REFERENCE]. The phenomenon of {\it Volunteered} Geographic Information (VGI) in particular was studied extensively since the term was coined \cite{Goodchild:2007vt}. Developing understanding of VGI was made possible by the success of services 

[ELENA: NEED TO ADD DESCRIPTION OF PAST ATTEMPTS AT DOING THIS SUCCESSFULLY]

\subsection{Open data and open Geographic Information}

GI is often associated to open data: data that anyone can access, use and share\footnote{\url{http://theodi.org/faq}.}. 

Government-funded NMCAs are natural owners of geospatial data and are commonly expected to release it in the open for the public good. In Great Britain, for example, since 2015 the local NMCA Ordnance Survey\footnote{See \url{http://www.ordnancesurvey.co.uk/}.} has opened a substantial volume of data that was previously available to the public as commercial products only, e.g. "Open Names", a place-name index, and "Open Roads", the generalised geometry and network connectivity of the road network.
	
The availability of such high quality and authoritative sources becomes a substantial enabler for the creation of new, original geospatial data. There where GI could only be created from scratch - as in OpenStreetMap's case when it was started in the UK in 2007 - it is now possible to rather complementing and augmenting what is already available.

All useful and needed geospatial data is not always made available in the open, though. Demand and offer for open data can be suppressed, for example, due to failure in recognising open data-enabled business models, restrictive legislative and patent systems, or charging for public datasets \cite{shadboltpaf}. The OLAF problem, described later in this document, is a great example of this.
	
One of the assumption at the base of our research is that the people's contribution is useful and necessary to work around this blockages. People can be enabled, through technology, to capture and curate data, alongside what is already published by governments and businesses. Human-machine hybrid systems such as the one proposed in this paper can be built to facilitate this process.

\subsection{Challenges of crowdsourcing geospatial data}

[SUMMARY OF LITERATURE ON THIS SUBJECT]

\subsection{The Open Legal Address File (OLAF)}
\label{subs:the-problem-of-creating-an-olaf}

\textbf{PAF} The main use case for this research is creating an Open Legal Address File (OLAF) for the UK: a dataset that is functionally equivalent to the "Postcode Address File" or "PAF"\footnote{PAF is a registered trademark by Royal Mail plc. For convenience we won't show the registered trademark sign "\textregistered" in this document every time we refer to it.}: a commercial dataset that lists all known valid addresses and postcodes for the UK and was created over many years of operations by the once state-owned mail delivery services company Royal Mail.

An act of law\footnote{The Postal Service Act 2000, part VII, article 116 \cite{postalserviceact2000}.} makes PAF ownership of Royal Mail and requires them to make PAF available on "reasonable terms" to "any person who wishes to use it". To this day, though, the "reasonable terms" did not translate into making PAF open data. The dataset is available from Royal Mail and its resellers only as a commercial product. This is considered an anomaly by many, as "reasonable terms for the external use of PAF data by third parties should be no more than the marginal cost of distribution (...)" \cite{odugresponse}. 

In October 2013 Royal Mail and its assets were privatised\footnote{Royal Mail is currently a public limited company with only a 30\% of shares controlled by the Government. The Postal Services Act 2011 plans to reduce the Government control down to 10\% \cite{postalserviceact2011}.}, including PAF. The UK Parliament House of Commons' Public Administration Select Committee called the sale of PAF "unacceptable and unnecessary" and recognised that PAF's "disposal for a short-term gain will impede economic innovation and growth" \cite{pascod}.

The opportunity to create an alternative to PAF is an ideal case study. The term "legal addresses" in OLAF refers to all addresses that are, by law, in the public domain, hence have no restrictions in terms of intellectual property or privacy protection and can be published as open data\footnote{Legal addresses belong to either or both of the following two categories: a) they are the addresses of current or past UK residents who are or were registered on the public electoral register, and b) they are the addresses of past or present UK companies that are or were registered at the relevant British registrar, such as Companies House for England and Wales.}.

\textbf{Problem specifications} It is useful to give a high level specification of OLAF to clarify what the expected end result is. 

OLAF is a list of valid UK addresses. For the objectives of this study, and without going into the detail of official standards such as BS7666\footnote{See \url{http://shop.bsigroup.com/ProductDetail/?pid=000000000030127201}.}, it is possible to give a practical, operational definition of what a "valid UK address" is, that is, in addition to the name of the addressee, {\it the set of unambiguous information needed to instruct any delivery service operator to deliver a piece of mail at a given public address (a private house, a commercial establishment etc.), from a consumer perspective}. 

An address has the following properties\footnote{These are actually a subset of BS7666, that includes elements consumers are not required to specify in an address such as the UPRN: a nationally unique number assigned by the UK National Land \& Property Gazetteer to local authorities in order to give a unique identity to any addressable piece of land or property. UPRNs are also not available as open data.}:

\begin{enumerate}
    \item A "primary addressable object name" (PAON), e.g. "10". The PAON identifies one property.
    \item A "secondary addressable object name" (SAON), e.g. "Flat 2". The SAON identifies a part of a property and is optional.
    \item A street name, e.g. "Downing Street".
    \item A town name, e.g. "London".
    \item An UK postcode, e.g. "SW1A 2AA".
\end{enumerate}

For simplicity, note that the terms "place", "road" or "street" are used interchangeably here and in the rest of this document, as they are equivalent from an OLAF perspective. The "street name" in the model above, for example, may be the name of a square, etc.

The elements of an address are not case sensitive and tolerant of abbreviations and alternative spellings and punctuation used in common British English language (e.g "Downing Street" is the same as "DOWNING ST.").

[ELENA: ADD A DESCRIPTION OF THE CHALLENGES OF THIS SPECIFIC PROBLEM]

