\documentclass{llncs}

\usepackage{pbox}
\usepackage[utf8]{inputenc}
\usepackage{url}
%% This was made necessary to break` long URLs, no other system I tried worked and a few URLs still fail; see http://tex.stackexchange.com/a/10419/60582 
\makeatletter
\g@addto@macro{\UrlBreaks}{\UrlOrds}
\makeatother
\usepackage{graphicx}
\graphicspath{{./images/}}

\title{Using hybrid human-machine workflows to create geospatial data}

\author{AUTHOR 1\inst{1}, AUTHOR 2\inst{1} \and AUTHOR 3\inst{2}}
\institute{INSTITUTE 1 \email{EMAIL FOR AUTHOR 1} \and INSTITUTE 2}

\date{December 2015}

\begin{document}

\maketitle

\begin{abstract}
As more open data is published by governments and organisations, the task of creating new data evolves from being a ground-up process - where data is collected and shaped from scratch - to an enhancement process, where pre-existing sources and original additions merge into new, valuable data products. 

Machine learning, computer vision and automation in general can't (yet?) address all challenges in this area, and available data, computation and original contributions by human participants - e.g. through crowdsourcing - need integration and enabling each other through hybrid human-machine workflows. 

In this paper we present a possible application of this model, and setup a platform to deal with a real world use case: the creation of "OLAF", an open dataset of all valid UK addresses, starting from available open geospatial data and augmenting it through computational inference and crowdsourcing. Experimental evaluation show the feasibility and effectiveness of the approach.
\end{abstract}

\begin{keywords}
Crowdsourcing, Geographic Information, hybrid human-machine systems, open data 
\end{keywords}

\begin{itemize}
    \item MAX 18 PAGES
    \item OPEN POINTS
        \begin{itemize}
            \item RESEARCH QUESTION, DESCRIPTION AND ANALYSIS OF THE FINAL EXPERIMENTS ARE STILL MISSING. WHAT CAN BE DONE THAT IS INTERESTING, RELEVANT TO RESEARCH AND FITS THE TIME AVAILABLE?
        \end{itemize}
    \item WEAKNESSES 
        \begin{itemize}
            \item STRONG FOCUS ON USE CASE, POOR GENERALISATION, BUT IS IT NECESSARY?, GIVEN THE CONFERENCE'S EXPLICIT INTEREST IN "Use cases and experiences with human computing and crowdsourcing applications" \url{http://icwe2016.inf.usi.ch/topics#crowd}
            \item ARGUABLE CHOICE OF NOT USING GOLD TESTS
            \item I LIKE THE IDEA OF THE "SOCIAL MACHINE MIX" I PRESENT, BUT THAT SHOULD PROBABLY BE A PAPER OF ITS OWN, UNLESS I CAN FIND SOMEONE WHO DEVELOPED THE SUBJECT BEFORE ME
            \item WON'T HAVE TIME TO DEVELOP ROBUST STATISTICAL DISCUSSION OF THE RESULT, E.G. EXPECTED DISTRIBUTION OF THE WORKER SUBMISSIONS, POSSIBILITY TO USE THAT TO DETECT ANOMALIES ETC., LOTS OF POTENTIAL THERE
            \item I STILL FEEL LIKE I DON'T HAVE A SUFFICIENT AWARENESS OF LITERATURE AROUND ALL SUBJECTS I TOUCH, THERE COULD BE PAST WORK ABOUT ALMOST EVERYTHING I DISCUSS
            \item QUANTITATIVE PROOF OF THE INFERENCE ALGORITHMS BEING EFFECTIVE (WHAT I WANTED FROM CHECKING THE ORDNANCE SURVEY DATA)
        \end{itemize}
\end{itemize}

\section{Introduction}

\subsection{Crowdsourcing}

    Crowdsourcing is {[}...{]}. More generally {[}SOME LIGHTWEIGHT REFERENCE TO WHAT A SOCIAL MACHINE IS{]} [{[}BLAH BLAH SOCIAL MACHINES AS POSSIBLY ONE OF THE ONLY WAYS TO SOLVE PROBLEMS LIKE THIS + SOME EXCUSE TO CITE \cite{OReilly:2015uo}{]}

\subsection{Crowdsourcing Geographic Information}

    Among the many applications of crowdsourcing is the collection and maintenance of geospatial data, or "Geographic Information" (GI).
    
    Although it is commonly recognised for GI to have a significant economic and social value \cite{Sui:2012uf}[THIS IS A REFERENCE TO AN ENTIRE BOOK, LIKELY UNSUITABLE], the effort national mapping and cadastre agencies (NMCAs) worldwide put into producing and updating cartography has been in decline for several decades \cite{ESTES:1994vz}. In the U.S., for example, the Geological Survey (USGS) no longer attempts to update its maps on a regular basis and the National Research Council promotes a vision in which {[}...{]} \cite{Committee:1993vp}.
    
    {[}Somewhere in the literature someone said that the emergence of VGI was actually suggested by one of those authorities{]}
    
    Crowdsourcing GI is then seen as the cost effective and "good enough" solution to this problem [CITATION OF SOME SCHOLAR SAYING THAT GOOD ENOUGH IS... GOOD ENOUGH]. The phenomenon of {\it Volunteered} Geographic Information (VGI) in particular was studied extensively since the term was coined \cite{Goodchild:2007vt}. Developing understanding of VGI was made possible by the success of services such as Wikimapia\footnote{\url{http://wikimapia.org/}.} or OpenStreetMap\footnote{\url{http://www.openstreetmap.org/}.}. The latter is likely the best known VGI-based mapping service available today.

\subsection{Open data and open Geographic Information}

    Open data is data that anyone can access, use and share\footnote{\url{http://theodi.org/faq}.}. 
    
    NMCAs, as governments and private organisations, are becoming more sensitive to the opportunities arising from publishing and re-using open data. In Great Britain, for example, since 2015 the local NMCA Ordnance Survey\footnote{\url{http://www.ordnancesurvey.co.uk/}.} has released in the open a substantial volume of data that was previously available to the public as commercial products only, e.g. "Open Names"\footnote{\url{https://www.ordnancesurvey.co.uk/business-and-government/products/os-open-names.html}.}, a place-name index, and "Open Roads"\footnote{\url{https://www.ordnancesurvey.co.uk/business-and-government/products/os-open-roads.html}.}, the generalised geometry and network connectivity of the road network.
    
    The availability of such high quality and authoritative sources becomes a substantial enabler for the creation of new, original geospatial data. There where GI could only be created from scratch - as in OpenStreetMap's case when it was started in the UK in 2007 - it is now possible to rather focus the effort of the crowd on complementing what is already available, or augmenting it.

\subsection{Crowdsourcing open data}

    The work described in this paper took place in the context of a larger research programme aimed at assessing the feasibility of building original open data by using non-expert human contributions, technology systems and, where available, other pre-existing open data. 
    
    Although most of the effort of producing and releasing open data is commonly expected of governments and businesses, there are industry sectors and domains of knowledge where resistance to change can halt or substantially slow down this process. Demand and offer for open data is strongly suppressed, for example, due to failure in recognising open data-enabled business models, restrictive legislative and patent systems, or charging for public datasets \cite{shadboltpaf}. 
    
    The assumption at the base of our research is that the people's contribution is necessary to address this problem. People can be enabled, through technology, to capture and curate data, alongside what is already published by governments and businesses. Crowdsourcing is just one of the formulae by which socio-technical systems can be built for this purpose. This contribution is instrumental to delivering and curating the open data needed to build and operate a comprehensive {\it national information infrastructure} (NII).

\subsection{Challenges of crowdsourcing geospatial data}

    Two groups of challenges are relevant to the availability of good quality geospatial open data: on one side, exploiting it to improve the quality and enhance the functionality of pre-existing VGI initiatives, and, on the other, unleash completely new products and services.
    
    [CHALLENGE OF MAINTENANCE, AFTER COLLECTION]
    
    [SHOULD I WRITE MORE ABOUT THE OTHER CHALLENGES, TOO?]
    
    Completeness is a key element in the quality of geospatial data. In this paper we propose a method to improve the coverage of existing geospatial data where - for any reason - it is not possible to get volunteers to survey one specific geographical area or provide more detail than what is available already. This was described, for example, by **** , observing how OpenStreetMap volunteers naturally tend to avoid ***.
    
    {[}
    Add
    \begin{itemize}
    	\item {[}rationale of why we thought this was relevant, some justification in literature review{]}
    	\item {[}novelty of what we propose{]}
    \end{itemize}
    {]}
    
    {(}...{)}

\section{Background}

\subsection{The Open Legal Address File}

    The main use case for this research is the attempt of creating a new geospatial dataset that is functionally equivalent to the "Postcode Address File" or "PAF"\footnote{PAF is a registered trademark by Royal Mail plc. For convenience we won't show the registered trademark sign "\textregistered" in this document every time we refer to it.}: a commercial dataset listing all known valid addresses and postcodes for the UK at a given point in time. 
    An act of law\footnote{The Postal Service Act 2000, part VII, article 116 \cite{postalserviceact2000}.} makes PAF ownership of the Royal Mail: the ex-postal service monopolist in the UK, now a public limited company with only a 30\% of shares controlled by the Government\footnote{The Postal Services Act 2011 plans to reduce the Government control down to 10\% \cite{postalserviceact2011}.}. The same act of law requires Royal Mail to make PAF available on "reasonable terms" to "any person who wishes to use it".
    
    To this day, though, the "reasonable terms" translated only into PAF being made available by Royal Mail and its resellers as a commercial product\footnote{As an end user who wants to access PAF's data one may incur Royal Mail licence fees going from \pounds0.012 per transaction on a public website to \pounds90,000 per year for unlimited internal use by a corporate group. Even charitable organisation may not take advantage of free PAF access unless their income is less than \pounds10m/year.}. This is considered an anomaly by many, as "reasonable terms for the external use of PAF data by third parties should be no more than the marginal cost of distribution (...)" \cite{odugresponse}. 

    To further limit the opportunities for PAF to be made open, in October 2013 Royal Mail and its assets were privatised, including PAF\footnote{\url{http://www.theguardian.com/uk-news/2014/mar/17/royal-mail-privatisation-ministers-rebuked-selling-data}}. The UK Parliament House of Commons' Public Administration Select Committee, just a few month later, called this "unacceptable and unnecessary" and recognised that PAF's "disposal for a short-term gain will impede economic innovation and growth" \cite{pascod}.

    This makes the opportunity of creating an alternative to PAF an ideal case study. We will call this the "Open Legal Address File", or "OLAF". The term "legal addresses" refers to all addresses that are, by law, in the public domain, hence have no restrictions in terms of intellectual property or privacy protection and can be published as open data\footnote{Legal addresses belong to either or both of the following two categories: a) they are the addresses of current or past UK residents who are or were registered on the public electoral register, and b) they are the addresses of past or present UK companies that are or were registered at the relevant British registrar, such as Companies House for England and Wales.}.

    {[}Definition of address{]}
    
    Many complementary and/or alternative strategies are possible and need to be used jointly to build OLAF. Among these is the opportunity to re-use published open datasets of addresses so to infer the existence of more addresses. 
    
    E.g. it is intuitive that if some source refers to the existence of house numbers 3, 5 and 9 in some street, and all are associated to the same postcode, it is very likely that number 7 exists as well and is associated to the same postcode\footnote{House numbering and postcode association are heavily dependant of the conventions used in the country the problem is applied to. This paper always refers to the UK conventions.}. It is shown experimentally that the method is effective, as it can produce large volumes of addresses from available open data\footnote{This was tested against the single largest known source of addresses open data for England and Wales: Land Registry's "Price Paid Data". See \url{https://www.gov.uk/government/collections/price-paid-data}.}.
    
    The experiments described in this paper implement the above strategy only, and  use crowdsourcing to validate sets of inferred addresses, as participants are asked to virtually survey the streets using pictures sourced from Google Street View.
    
    It has to be noted that not all existing house numbers are visible by surveying a street. E.g. there are no obligations in the UK to affix a house number or house name sign. Moreover, some of the house numbers may be associated to dwellings that are not visible unless the property is accessed, beyond what Google Street View's photos can capture.
    
    {(}...{)}
    
[ELENA: ADD A DESCRIPTION OF THE CHALLENGES OF THIS SPECIFIC PROBLEM]

\input{3_approach.tex}
\section{Experiment design}

[PLEASE NOTE: THE SECOND EXPERIMENT DESCRIBED BELOW IS NOT WORTH DOING, ONCE YOU DO THE MATH. IT WOULD REQUIRE AN EXCEPTIONALLY HIGH MAJORITY CONSENSUS ON CROWDFLOWER (~94\%) TO ASSURE EQUIVALENCE TO THE TARGET FLEISS] 

\subsection{Research hypothesis}

Our work was centred around the hypothesis below:

\begin{enumerate}

    \item Paid crowdsourcing can be used to address some of the limitations of VGI, e.g. to survey streets on demand, quickly and reliably without the need to wait for a volunteer to offer herself for the task.
    
    \item Paid crowdsourcing can be effective at image labelling tasks that are more complex than what is commonly found in literature, as in the case of finding house numbers - or the lack thereof - in interactive visualisations of streets as in Google Street View.

    \item Effective workflow systems can be implemented by using just the native functionality of third party crowdsourcing services in the cloud such as CrowdFlower, hence maximising the system scalability and availability in respect to deploying custom components (code, application and database servers etc.)

    \item An iterative crowdsourcing workflow is more performing...
    
\end{enumerate}

To test these hypothesis, we carried out two experiments. Workers were offered to browse a street through a combination of Google Maps and Google Street View and asked to identify the lowest and the highest house numbers. The systems implementing the experiments were fully hosted on CrowdFlower, and only Google and CrowdFlower's native functionality was used, complemented by some "offline" scripting aimed at pre-processing the dataset and consolidating the results.

To test the first and second hypothesis, the potential for the community of Workers to converge on consensus on some of the streets was assessed, targeting a high degree of statistical confidence on the results.

To test the fourth hypothesis, two different approaches to judgement collection were implemented and their performance compared, both aiming at achieving results of equivalent quality.

The systems of the two experiments were indistinguishable from a Worker point of view. Participants used the same user interface, were paid the same amount etc.

\subsection{Dataset}

For our experiments, we used all roads within scope - as described earlier in this document - for which no other tool but surveying could be used to produce the missing house numbers. These are those roads that did not offer sufficient data in LRPP to apply the inference algorithms. This comprises of 184 streets out of the 1,949 in the original sample. LRPP providing little or no data about these streets over 20 years of records suggests that these are "difficult streets", with few or no buildings, and likely rural.

To simulate a real-world scenario in which different streets offer different degrees of interest to the Requester, fictional prioritisation criteria were associated to each of the five areas in scope. The objective was engaging the Workers on the highest prioritisation street first, possibly not completing coverage for the lowest prioritisation streets if budget was limited\footnote{The example scenario that was used as reference was the one where a charitable organisation wanted to run a campaign to support families in deprived areas in London. In order to produce the missing addresses, the charity would naturally prioritise their work by using the deprivation open data published as part of the latest census by the UK Office for National Statistics, e.g. at \url{https://www.nomisweb.co.uk/census/2011/qs119ew}.}.

The data and scripts used to prepare the experiment are available at \url{https://github.com/Digital-Contraptions-Imaginarium/OLAF-yr2_lab}.

\subsection{Evaluation metrics}

We evaluate the performance of the approach as the average total cost per road required to reach Worker consensus. The lower the cost, the more effective the approach.

\subsection{Experimental conditions}

\textbf{Experiment 1} {\it Judgement collection: iterative, 5 judgements per road. Source dataset: the full sample of 184 roads, in "rounds" of max 10 roads depending on availability, chosen according to prioritisation. As road reach consensus, they are removed from the following round to make room for more roads. Stop condition: either (i) a total 400 judgements is collected or, (ii) consensus is reached on all roads, or (iii) at completion of any iteration, the number of total duplicate judgements by reliable Workers is higher than the number of unique judgement, whichever condition is verified first. Consensus: Fleiss kappa of 0.6 on roads where house numbers could be found, 0.8 on roads where house numbers could not be found, calculated at completion of each iteration.} 

\textbf{Experiment 2} {\it Judgement collection: one-shot, with a minimum of **** judgements and a maximum of **** per road. Collection is stopped on a road by road basis when consensus is reached. Source dataset: the same set roads that were subject to judgement in any round of Experiment 1. Stop condition: either (i) the same total number of judgements of Experiment 1 is collected, or (ii) consensus is reached on all roads, whichever condition is verified first. Consensus: simple agreement \% of Workers equivalent or superior to the worst case scenario Fleiss kappa of Experiment 1, calculated at completion of each judgement\footnote{CrowdFlower provides natively the functionality to continue judgement collection on each road until a target simple agreement x\% between Workers is achieved. Differently than Fleiss' kappa that was used in experiment 1, this does not take into consideration the distribution of judgements in disagreement. It is possible, though, to calculate a simple agreement \% that is statistically equivalent or better, for each combination of number of judgements and target kappa. E.g. the smallest simple agreement \% that can achieve a Fleiss' kappa of 0.8 on 30 judgements is 90\%, that is like saying that 27 Workers were in agreement, whatever the judgements were by the other workers.}.} 

\input{5_results.tex}
% \section{Literature review}
\input{6_discussion_and_conclusion.tex}

\textbf{Acknowledgements.} [WRITE THIS, NOTHING REVEALING IF THE REVIEW IS BLIND]

\documentclass{article}
\usepackage{url}
\begin{document}
\nocite{*}
\bibliography{main}
\bibliographystyle{splncs03}
\end{document}


\end{document}

