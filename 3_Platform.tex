\section{Platform}

\subsection{overview: short summary of what the platform does}

\subsection{Core design principles}

In designing the platform, we tried to achieve a good combination of the design principles described below:

\textbf{Support to human-machine workflows} The need to rely on hybrid human-machine workflows was a key assumption in our research. The solution was required to automate as much as possible of the data creation / curation process while effectively integrating human components, from the crowdsourcing of part of the data to the administration of the process itself.

\textbf{Data re-use} Making the best possible use of pre-existing, reliable data was a key design driver, in particular to take advantage of the substantial volume of open data published in the UK over the last two-three years, and expecting a similar trend to be observable in many other countries in the upcoming years.

\textbf{Open source software} It was assumed that any software component required by the solution could be built using open source software only, also to preserve any available budget for the compensation of paid contributors to the crowdsourcing campaigns.

\textbf{Scalability and high availability} The solution was designed to be highly scalable and available, beyond the needs of the experiments described later in this document and suitable for real-world deployment.

\textbf{Versatility} The solution needed being versatile and suitable to support different workflows, input data sources and target output datasets. Despite the extensive literature, crowdsourcing in particular is still more of a craft than science and no design formula can assure success without experimentation. The design of the solution needed to support different forms of crowdsourcing across its many dimensions: paid contributors vs volunteers, results aggregation, quality assessment etc.

\subsection{introduce the platform according to the dimensions I use in my presentations. Note that this is a description of the platform as we envision it for the future, not as bespoke tool to elicit house numbers}

what/who/how

what to crowdsource (inputs/outputs/creation vs validation tasks)
who is the crowd 
how: task breakdown, open vs closed answers, incentives etc etc etc

\subsection{implementation}