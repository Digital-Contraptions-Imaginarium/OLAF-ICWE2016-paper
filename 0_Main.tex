\documentclass{llncs}

\usepackage{floatrow}
\usepackage[utf8]{inputenc}
\usepackage{url}
%% This was made necessary to break` long URLs, no other system I tried worked and a few URLs still fail; see http://tex.stackexchange.com/a/10419/60582 
\makeatletter
\g@addto@macro{\UrlBreaks}{\UrlOrds}
\makeatother
\usepackage{graphicx}
\graphicspath{{./images/}}

\title{Using hybrid human-machine workflows to create geospatial data}

\author{AUTHOR 1\inst{1}, AUTHOR 2\inst{1} \and AUTHOR 3\inst{2}}
\institute{INSTITUTE 1 \email{EMAIL FOR AUTHOR 1} \and INSTITUTE 2}

\date{December 2015}

\begin{document}

\maketitle

\begin{abstract}
As more open data is published by governments and organisations, the task of creating new data evolves from being a ground-up process - where data is collected and shaped from scratch - to an enhancement process, where pre-existing sources and original additions merge into new, valuable data products. 

Machine learning, computer vision and automation in general can't (yet?) address all challenges in this area, and available data, computation and original contributions by human participants - e.g. through crowdsourcing - need integration and enabling each other through hybrid human-machine workflows. 

In this paper we present an experimental platform that implements this model, aimed specifically at the creation of geospatial data. The platform is then applied to a real world use case: the creation of "OLAF", an open dataset of all valid UK addresses, starting from available open data and augmenting it through computational inference and microtask crowdsourcing. Experimental evaluation show the feasibility and effectiveness of the approach.
\end{abstract}

\begin{keywords}
Crowdsourcing, Geographic Information, hybrid human-machine systems, open data 
\end{keywords}

\begin{itemize}
    \item MAX 18 PAGES
    \item OPEN POINTS
        \begin{itemize}
            \item XXX
        \end{itemize}
    \item WEAKNESSES 
        \begin{itemize}
            \item STRONG FOCUS ON USE CASE, POOR GENERALISATION, BUT IS IT NECESSARY?, GIVEN THE CONFERENCE'S EXPLICIT INTEREST IN "Use cases and experiences with human computing and crowdsourcing applications" \url{http://icwe2016.inf.usi.ch/topics#crowd}
        \end{itemize}
\end{itemize}

\section{Introduction}

[ELENA'S SUGGESTED FEEL FOR THIS SECTION IS:] geospatial data is a critical component of many applications. Yet not all of it is readily available, and when available, its use is associated with high costs. We look at crowdsourcing, in particular microtask crowdsourcing, as a means to offer geospatial data for the public good. We focus on a specific use case, the OLAF, which has received a lot of attention in the UK and propose a platform to outsource OLAF microtasks to a crowd. The platform can be configured to support different types of tasks and participation models (paid, unpaid, individual vs team etc) and is available as open source. We evaluate it on an experiment ... We believe this could be a useful tool for geospatial researchers and practitioners interested in systematically using crowdsourcing in their daily work.

\subsection{Crowdsourcing}

    Crowdsourcing is {[}...{]}. More generally {[}SOME LIGHTWEIGHT REFERENCE TO WHAT A SOCIAL MACHINE IS{]} [{[}BLAH BLAH SOCIAL MACHINES AS POSSIBLY ONE OF THE ONLY WAYS TO SOLVE PROBLEMS LIKE THIS + SOME EXCUSE TO CITE \cite{OReilly:2015uo}{]}


\section{Background and related work}

\subsection{Crowdsourcing geospatial information}

Geographical Information Systems (GIS) are extensively studied in literature. They are systems designed to capture, manipulate, analyse and visualise geographical data, that traditionally is produced and maintained by professional organisations or institutions such as the NMCAs. 

In the early 2000's, the consolidation of web technologies permitted the emergence of publicly available, web-based GIS systems such as OpenStreetMap\footnote{Created in 2004, see \url{https://www.openstreetmap.org}.} ("OSM" in the following), Google Maps\footnote{Launched in 2005, see \url{https://www.google.com/maps}. Google Earth is actually precedent to Google Maps and was launched in 2001, but at the time it was available as a downloadable desktop application only, see \url{https://www.google.co.uk/intl/en_uk/earth/explore/products/desktop.html}.} and Wikimapia\footnote{Created in 2006, see \url{http://wikimapia.org/}.}. At the same time, non-GIS applications started integrating geospatial data into their features, too; e.g. photo publishing platform Flickr\footnote{See \url{https://www.flickr.com/}.} permitted its users to upload and locate their media on interactive maps. 

Not only GIS technology was made available to the larger public, but laypersons could start contribute to the systems, either augmenting pre-existing data or creating new data, in a transformation that some called "the democratisation of GIS" \cite{Butler:2006fe}. This was around the same time the term "crowdsourcing" itself was first used\footnote{See \url{http://www.wired.com/2006/06/crowds/}.}.

Crowdsourcing geospatial data started being the subject of extensive study. In 2007 Michael F. Goodchild examined the phenomenon as a new form of citizen science and coined the term "Volunteered Geographic Information" (Volunteered GI, or VGI) in his "Citizens as sensors: the world of volunteered geography" article \cite{Goodchild:2007vt}. In the same paper, Goodchild started examining the contributors' motivation and also proposed the hypothesis that relying on the crowd could be much more cost-effective at maintaining geospatial data than any of the traditional ways.

Almost ten year later [SOMETHING TO PUT AS A CLOSURE, HIGHLIGHT HOW GIS IS PART OF MOST WEB USERS' EXPERIENCE BLAH BLAH] 

 \subsection{Challenges of crowdsourcing geospatial data}

The challenges of crowdsourcing geospatial data are extensively discussed in literature, but a few in particularly were addressed by our research and are summarised in the following.

\textbf{Contributor credibility} Both VGI and paid crowdsourcing share the problem of the contributors' credibility. This was examined for example in \cite{Flanagin:2008ck}. Although Flanagin and Metzger's work is focused on VGI only, crowdsourcing GI in general shares the same problems of data gatekeeping, quality control and credibility in general of the contributors and their output. 

Harvey \cite{Harvey:2012wsa} also highlights a distinction between VGI and GI that is instead collected as a side-effect of using a service, typically mobile networks, or GPS-enabled applications, that he calls "contributed" GI. Harvey argues that understanding the nature of the sources of crowdsourced data is instrumental to identify possible biases. As an example, he offers a scenario where a local activist group draws on OSM data to show the location of squats, only to find out that the data has been volunteered by the police force. 

\textbf{Data reliability} Even assuming that the contributors are sufficiently credible, the quality of their work may anyway not be sufficiently reliable for the target application. Haklay in \cite{Haklay:2010vs} performed a comparative study of the data offered by OSM and the British NMCA Ordnance Survey ("OS" in the following). Already in 2008, he found that OSM offered a reasonable accuracy of about 6m and an overlap of up to 100\% of roads in OS' data. 

Moreover, although the quality of the data was inconsistent and heavily dependent on the contributor rather than the complexity of the spatial unit, further research in  \cite{Haklay:2010wf} suggested that, the volume of contributors - even when working without a central coordination - can be considered as an intrinsic quality assurance measure, in a way analogous to what happens with open source software.

Crowdsourcing GI is seen a the cost effective and "good enough" solution to [SOMETHING] [REFERENCE]. 

\textbf{Data completeness} Volunteered GI in particular also has a completeness issue. Haklay attributes this to the loose organisation of the contributors \cite{Haklay:2010vs} [BLAH BLAH] and socioeconomic barriers: at the time when physical survey of the streets was still one of the main channels to create new data, OSM creator Steve Coast stated in an interview that "Nobody wants to do council estates" \cite{Anonymous:2007ux} and in general there will always be places volunteers won't be interested in contributing to.

Moreover, even if completeness could be achieved eventually, depending on the application the data may be useful only if it is timely available.

\textbf{Motivation and incentives} The volume of literature on VGI in respect to non-volunteered crowdsourcing of GI appears to represent some degree of mistrust in the possibility that the latter kind of system could be successful, or worth exploring in general in respect to traditional practices for geospatial data creation and maintenance.   

[WHERE TO PUT THIS? PAID CROWDSOURCING CAN BE FASTER THAN VGI AND RELYING ON A CROWD THAT IS DETACHED BY THE PLACES CAN ASSURE THE LACK OF BIAS]

\subsection{Open data and open Geographic Information}

Because of the extensive research work done on systems that leverage volunteer contribution and are designed for the pubic good, GI is often implicitly associated to open data: "data that anyone can access, use and share"\footnote{\url{http://theodi.org/faq}.}. 

Government-funded NMCAs are natural owners of geospatial data and are commonly expected to release it in the open for the public good. In Great Britain, for example, since 2015 the local NMCA Ordnance Survey\footnote{See \url{http://www.ordnancesurvey.co.uk/}.} has opened a substantial volume of data that was previously available to the public as commercial products only, e.g. "Open Names", a place-name index, and "Open Roads", the generalised geometry and network connectivity of the road network.
	
The availability of such high quality and authoritative sources becomes a substantial enabler for the creation of new, original geospatial data. There where GI could only be created from scratch - as in OSM's case when it was started - it is now possible to rather complementing and augmenting what is already available.

All useful and needed geospatial data is not always made available in the open, though. Demand and offer for open data can be suppressed, for example, due to failure in recognising open data-enabled business models, restrictive legislative and patent systems, or charging for public datasets \cite{shadboltpaf}. The OLAF problem, described later in this document, is a great example of this.
	
One of the assumptions at the base of our research is that the people's contribution is useful and necessary to work around this blockages. People can be enabled, through technology, to capture and curate data, alongside what is already published by governments and businesses. Human-machine hybrid systems such as the one proposed in this paper can be built to facilitate this process.

\subsection{The Open Legal Address File (OLAF)}
\label{subs:the-problem-of-creating-an-olaf}

\textbf{The Postcode Address File} The main use case for our research is creating an Open Legal Address File\footnote{The term "legal addresses" in OLAF refers to all addresses that are, by law, in the public domain, hence have no restrictions in terms of intellectual property or privacy protection and can be published as open data. Legal addresses belong to either or both of the following two categories: a) they are the addresses of current or past UK residents who are or were registered on the public electoral register, and b) they are the addresses of past or present UK companies that are or were registered at the relevant British registrar, such as Companies House for England and Wales.} for the UK: a dataset that is functionally equivalent, for example, to Royal Mail's "Postcode Address File" ("PAF"\footnote{PAF is a registered trademark by Royal Mail plc. For convenience we won't show the registered trademark sign "\textregistered" in this document every time we refer to it.}). PAF is a commercial dataset that lists all known valid addresses and postcodes for the UK and was created by the once state-owned mail delivery services company over many years of operations.

An act of law\footnote{The Postal Service Act 2000, part VII, article 116 \cite{postalserviceact2000}.} makes PAF ownership of Royal Mail and requires them to make PAF available on "reasonable terms" to "any person who wishes to use it". To this day, though, the "reasonable terms" did not translate into making PAF open data. The dataset is available from Royal Mail and its resellers only as a commercial product. This is considered an anomaly by many, as "reasonable terms for the external use of PAF data by third parties should be no more than the marginal cost of distribution (...)" \cite{odugresponse}. 

In October 2013 Royal Mail and its assets were privatised\footnote{Royal Mail is currently a public limited company with only a 30\% of shares controlled by the Government. The Postal Services Act 2011 \cite{postalserviceact2011} plans to reduce the Government control down to 10\%.}, including PAF. The last reasonable expectation that the PAF could be made open data was lost. The UK Parliament House of Commons' Public Administration Select Committee called the sale of PAF "unacceptable and unnecessary" and recognised that PAF's "disposal for a short-term gain will impede economic innovation and growth" \cite{pascod}. Trying fill the gap left in the UK open geospatial data offering by this critical dataset makes an ideal case study. 

\textbf{The OLAF data model} It is useful to give a high level specification of the OLAF data model to clarify what an ideal dataset would look like. OLAF is a flat list of valid UK addresses. For the objectives of this study, and without going into the detail of official standards such as BS7666\footnote{See \url{http://shop.bsigroup.com/ProductDetail/?pid=000000000030127201}.}, it is possible to give a practical, operational definition of what a "valid UK address" is, that is, in addition to the name of the addressee, {\it the set of unambiguous information needed to instruct any delivery service operator to deliver a piece of mail at a given public address (a private house, a commercial establishment etc.), from a consumer perspective}. 

An address is characterised by the following properties\footnote{These are actually a subset of BS7666, that includes elements consumers are not required to specify in an address such as the UPRN: a nationally unique number assigned by the UK National Land \& Property Gazetteer to local authorities in order to give a unique identity to any addressable piece of land or property. UPRNs are also not available as open data.}:

\begin{itemize}
    \item A "primary addressable object name" (PAON), e.g. "10". The PAON identifies one property.
    \item A "secondary addressable object name" (SAON), e.g. "Flat 2". The SAON identifies a part of a property and is optional.
    \item A street name\footnote{For simplicity, note that the terms "place", "road" or "street" are used interchangeably here and in the rest of this document, as they are equivalent from an OLAF perspective. The "street name" in the model above, for example, may be the name of a square, etc.}, e.g. "Downing Street".
    \item A town name, e.g. "London".
    \item An UK postcode, e.g. "SW1A 2AA".
\end{itemize}

The elements of an address are not case sensitive and tolerant of abbreviations and alternative spellings and punctuation used in common British English language (e.g "Downing Street" is the same as "DOWNING ST.").

[ELENA: ADD A DESCRIPTION OF THE CHALLENGES OF THIS SPECIFIC PROBLEM]


\section{Platform}

\subsection{overview: short summary of what the platform does}

\subsection{Core design principles}

In designing the platform, we tried to achieve a good combination of the design principles described below:

\textbf{Support to human-machine workflows} The need to rely on hybrid human-machine workflows was a key assumption in our research. The solution was required to automate as much as possible of the data creation / curation process while effectively integrating human components, from the crowdsourcing of part of the data to the administration of the process itself.

\textbf{Data re-use} Making the best possible use of pre-existing, reliable data was a key design driver, in particular to take advantage of the substantial volume of open data published in the UK over the last two-three years, and expecting a similar trend to be observable in many other countries in the upcoming years.

\textbf{Open source software} It was assumed that any software component required by the solution could be built using open source software only, also to preserve any available budget for the compensation of paid contributors to the crowdsourcing campaigns.

\textbf{Scalability and high availability} The solution was designed to be highly scalable and available, beyond the needs of the experiments described later in this document and suitable for real-world deployment.

\textbf{Versatility} The solution needed being versatile and suitable to support different workflows, input data sources and target output datasets. Despite the extensive literature, crowdsourcing in particular is still more of a craft than science and no design formula can assure success without experimentation. The design of the solution needed to support different forms of crowdsourcing across its many dimensions: paid contributors vs volunteers, results aggregation, quality assessment etc.

\subsection{Crowdsourcing component... WHAT?}

[ELENA: introduce the platform according to the dimensions I use in my presentations. Note that this is a description of the platform as we envision it for the future, not as bespoke tool to elicit house numbers]

The following is a high level description of the crowdsourcing component of the platform according to the dimensions presented in \cite{Wearethedata:2015uo}. The description is independent of the specific objective the system could be deployed to address.

\textbf{What is outsourced} Because of the core design principles above, the scope of the outsourcing activity is centred around the production of new geospatial data or the correction / validation of pre-existing data, wherever the activity could not be performed but by engaging human agents. 

This typically translates into surveying the locations to observe some phenomenon and record the observation (e.g. "how many trees can be seen from longitude x and latitude y?"), or, alternatively, amend some previous recording of the same (e.g. "can you confirm that there is a hospital in Vicarage Rd, Watford, Hertfordshire?"). 

To fully leverage the crowd and in an attempt to maximise cost effectiveness, publicly available imagery of the locations is used rather than surveying the actual places. The performance of humans observers is expected to keep outperforming machine learning, computer vision and automation in general for a few more years, particularly in observing imagery that are ambiguous [IT WOULD BE NICE TO QUOTE SOMEONE HERE].

\textbf{Who is the crowd} VGI literature explored extensively how geospatial data can be created by crowds of intrinsically motivated volunteers, often taking responsibility of tasks that require substantial effort, such as surveying a street in the real world, including travel to get to the location, the use of equipment to record GPS tracks etc. 

Our platform was intentionally designed to explore the somehow opposite scenario. Our contributors are oblivious of the the context of the project, have no personal connection to the locations being surveyed, and likely find no motivation in contributing to the "cause" of open data in general. They perform survey {\it micro}tasks and are driven merely by the financial reward. This is typically the crowd we can recruit through the mainstream paid crowdsourcing platforms such as Amazon Mechanical Turk and Crowdflower. 

\textbf{How are the task oursourced}

\textbf{Why do people contribute}


what/who/how

what to crowdsource (inputs/outputs/creation vs validation tasks)
who is the crowd 
how: task breakdown, open vs closed answers, incentives etc etc etc

\subsection{implementation}
\section{Applying WODA to the Open Legal Address File problem}
\label{crowdsourcing-olaf}

\subsection{How primary open data sources shape the solution}

The availability of high quality and authoritative open data is central to the definition of the solution. An assessment of available sources highlighted OS' "Open Names"\footnote{Ordnance Survey is the national mapping agency for Great Britain. See \url{https://www.ordnancesurvey.co.uk/business-and-government/products/os-open-names.html}.}, (OSON) as suitable primary input. 

OSON lists place names, roads numbers and postcodes in Great Britain, but not (i) which house names and numbers are in which road, and (ii) which house names and numbers are associated with which postcode. Therefore, OLAF can be seen as an augmentation of OSON, obtained by adding those missing components.

The production of the missing data can be achieved through running three complementary processes described in figure \ref{fig:problem_decomposition_2} as {\it p1}, {\it p2}\footnote{98\% of UK addresses are characterised by a house number rather than a house name, so tackling {\it p1} is substantially more relevant to OLAF's completeness than {\it p2}.} and {\it p3}. 

\begin{figure}
	\includegraphics[width=1.0\textwidth]{problem-decomposition-2.png}
	\caption{A possible decomposition of OLAF}
	\label{fig:problem_decomposition_2}
\end{figure}

The partial availability of the missing data in other primary sources such as Land Registry's "Price Paid Data"\footnote{Land Registry is a non-ministerial UK Government department with the responsibility to register the ownership of land and property in England and Wales. See \url{https://www.gov.uk/government/collections/price-paid-data}.} (LRPP) enables to further refinement of process {\it p1} in four sub-processes {\it p1.1}, {\it p1.2}, {\it p1.3}\footnote{See \url{https://github.com/Digital-Contraptions-Imaginarium/OLAF-yr2_lab/blob/gh-pages/docs/README.md#error-in-inferred-house-numbers} for an assessment of the impact of error.} and {\it p1.4}. LRPP records every property ownership transfer in England and Wales since 1995, including their full addresses\footnote{LRPP is the largest open data source that include UK addresses.}, from which the inference of addresses is possible.

\subsection{The OLAF-specific workflow}

The generic workflow described in \ref{generic-workflow} can be specialised for OLAF by integrating {\it p1.1}, {\it p1.2} and {\it p1.4}. Figure \ref{fig:workflow_2} shows the new workflow and the mapping against the three processes. 

\begin{figure}
	\includegraphics[width=1.0\textwidth]{workflow-2.png}
	\caption{UML process diagram of the OLAF-specific workflow to create house numbers}
	\label{fig:workflow_2}
\end{figure}

\subsection{Ingestion of primary data sources} 

Both OSON and LRPP contain references to streets, by their name. The main challenge of harvesting LRPP is to identify unambiguously the streets the house numbers are associated to, across the two datasets. Issues such as differences in spelling (e.g. "Downing Street" instead of "Downing St") and association of the locations not to the same town, but to localities thereof (e.g. "Clapham" instead of "London"), need being managed with common data processing practices.

\subsection{House number inference} 
\label{inference-algorithms} 

\textbf{House numbering convention} Inferring house numbers is strictly dependent on the local numbering convention for the assignment of house numbers and names to buildings. In the UK, buildings are typically numbered sequentially starting from 1, at the extremity of the road closest to the town centre. Odd numbers are on the left-hand side, as seen from the town centre, while even number are on the right-hand side. House numbers can be suffixed by one or more letters: this is typical of larger buildings that at some point in time got divided into smaller dwellings. 
        
\textbf{Inference algorithms} Once the numbering convention is known, inference can be used to create a large volume of missing house numbers from the observation of known house numbers. Algorithms \ref{algo:inference-numbers} and \ref{algo:inference-numbers-suffix} below have a very high probability of correctly inferring the existence of house numbers\footnote{See \url{https://github.com/Digital-Contraptions-Imaginarium/OLAF-yr2_lab/blob/gh-pages/docs/README.md#exceptions-in-the-uk-house-numbering-system} for exceptions in the convention}. 

\vspace{5mm}

\begin{algorithm}[H]
    \KwData{The list of known house numbers in a road}
    \KwResult{The list of inferred house numbers in the same road}
    \eIf{the list includes at least one even and one odd number}{
        infer all numbers between the lowest and the highest known numbers\;
    }{
        \If{the list includes at least two even or two odd numbers}{
            infer all even/odd numbers between the lowest and the highest numbers\;
        }
    }
    \caption{Inference of house numbers}
    \label{algo:inference-numbers}
\end{algorithm}

\vspace{5mm}

\begin{algorithm}[H]
    \KwData{The list of known house numbers with suffixes in a road}
    \KwResult{The list of inferred house numbers with suffixes in the same road}
    \For{each house number appearing in the list with at least two suffixes}{
        infer all suffixes between the lowest and the highest known suffix, in alphabetical order\;    
    }
    \caption{Inference of house number with suffixes}
    \label{algo:inference-numbers-suffix}
\end{algorithm}

\subsection{Enabling inference for roads with no data} 

Inference of house numbers is enabled by one of the following two conditions: (a) the knowledge of two or more different house numbers in the same street and (b) the knowledge of two or more suffixes for the same number. The former has the highest potential to generate new house numbers.

When dealing with roads for which LRPP provides no data, it is useful to focus crowdsourcing on creating the conditions that enable that potential. To infer the largest sets of numbers it is necessary to use as input to algorithm \ref{algo:inference-numbers} the road's lowest and highest known house numbers\footnote{For simplicity, we did not consider (i) that it is also useful to know if the streets have both odd and even house numbers and (ii) the case where one house number only is known for a street, for which we could assume 1 to be the lowest house number.}.

\subsection{Stop conditions} 

The stop conditions referred to in the diagram are every condition not strictly related to the function of the system, e.g. the availability of budget. In other words, the workflow is iterated as long as budget is available, even if the aimed outcome is not achieved. 

\subsection{Preparation and augmentation for crowdsourcing} 

The data required by the crowdsourcing component may need preparation and augmentation before being used. In the case of OLAF, an additional primary data source is used to support the Workers in their tasks: OS' "Open Roads"\footnote{See \url{https://www.ordnancesurvey.co.uk/business-and-government/products/os-open-roads.html}.} (OSOR). 

OSOR is used to calculate the geographical coordinates of the extremities of the roads where the lowest and the highest house numbers are more likely to be found. These are offered as "points of interest" to support the crowdworkers in their search. It is an example of how open data can be used not necessarily as a direct input into producing the output dataset, but to support the human participants in their function, too.

\subsection{Data collection in CrowdFlower and assessment}

The crowdsourcing component of WODA can be configured to create the house numbers needed to enable inference. Observing the imagery of a street to identify its features is not conceptually different than crowdsourced labelling and annotation applications that are extensively studied in literature.\footnote{More detail is available at \url{https://github.com/Digital-Contraptions-Imaginarium/OLAF-yr2_lab/tree/gh-pages/docs#finding-house-numbers-as-a-labelling-exercise}.}

The following is a description of the approach that was used for crowdsourcing house numbers, that is common to all experimental conditions that were tested.

\subsubsection{Task model} \leavevmode \\ %% Why is this necessary to get a new line?

\textbf{Requester.} The Requester desires to gather the lowest and the highest house numbers that can be observed in a specified street, as they can be intelligibly identified by browsing imagery, or the lack thereof. The Requester requires the help of human agents to carry out the tasks, that we will call Workers in the following.

\textbf{Task.} Each HIT (Human Intelligence Task) consists of browsing the pictures of a street until achieving reasonable certainty of having identified the lowest and the highest house numbers or the lack thereof.

\textbf{Strategy.} 
The strategy relies on traditional crowdsourcing techniques for image labelling.

\textbf{Crowd $\rightarrow$ Worker.} Each Worker provides judgement on a task by browsing the pictures and declaring if she has found the lowest and the highest house numbers or none. Multiple Workers are asked to identify the house numbers for the same street. The resulting data is chosen through majority voting.

\textbf{Quality.} Quality is defined by a combination of (a) credibility of the Workers in responding to tests questions, and (b) consensus in the data submitted through repeated surveys of the same road. Aggregation takes place accordingly, as explained below.

\subsubsection{Workers quality}
    
Probing Workers using conventional test questions - e.g. where the correspondence of the Worker submissions is checked vs the same data collected by the research team as described in \cite{Kittur:2008gj} - would be a powerful tool to identify high vs low quality contributors, but is very expensive in OLAF's case. Spending a substantial part of the Worker's effort on test questions - e.g. making one out of three surveys a test - was not affordable.

As an alternative, though, simple test questions can be set up on data that is already available, in a way that is similar to classic anti-spamming techniques like CAPTCHAs as described in \cite{Difallah:2012ty}. In OLAF's case the name of the street itself is used: Workers are asked to copy and paste or type the name of the street as part of their survey. Workers that do not achieve the target accuracy are excluded from further work.

\subsubsection{Results aggregation}

Repeated surveys of the same road are equivalent to the use of repeated judgement in conventional image labelling exercises. This practice is described extensively in literature and demonstrate that the results produced by a few expensive expert individuals are comparable to what emerges from involving multiple answers by crowds of non-expert Workers, e.g. in \cite{Snow:2008wo} and \cite{Sheng:2008gra}. As the answers are inevitably noisy, different Workers were asked to survey the same road, and their responses are aggregated to decide what is the most likely and truthful observation. 
        
Approaches to aggregation are an equally well studied subject, and a majority decision is a natural option (e.g. \cite{Le:2010ug}). The detailed parameters and process of how consensus is defined and calculated are tuned for better performance and address issues specific to the context (e.g. in \cite{Hirth:2011fh}). 

In the case of OLAF, consensus is measured by using Fleiss' kappa statistics for inter-annotator agreement, as described for example in \cite{Nowak:2010gt}. For those streets where the house numbers {\it were} found, a kappa of 60\% on at least 5 surveys is sufficient consensus (e.g. see \cite{Landis:1977kv}). For those streets where the house numbers were {\it not} found, a kappa of 80\% on at least 10 surveys is required instead, as it is more likely that unreliable Workers agree in reporting that.

The number of 5 and 10 judgements is chosen because they are respectively the minimum number of judgements where 60\% and 80\% kappa can be achieved without the need of an unanimous agreement (4 vs 1 for 60\% and 9 vs 1 for 80\%). 

Rounds of 5 surveys per road are performed until consensus is reached on both its lowest and highest house numbers. Because of the nature of the task, new surveys are performed even if consensus is reached already on either of the two numbers.  

\subsubsection{Recruitment}

We sourced all our Workers from CrowdFlower. For each experiment, we created one dedicated CrowdFlower job. We used identical settings for each experiment set, consisting of the following parameters:

\textbf{Geography} Limited to the top 10 contributor countries in CrowdFlower where English is an official or officially recognised language\footnote{See \url{https://success.crowdflower.com/hc/en-us/articles/202703345-Crowd-Demographics}, the identification of the countries was last repeated on 19 December 2015, before the running the experiments described in this paper. The list of countries is: Bangladesh, Canada, India, Malaysia, Netherlands, Pakistan, Philippines, Sri Lanka, United Kingdom and United States of America.}.

\textbf{Skills} We chose Workers from the default CrowdFlower performance category (formerly named "level 2"), that accounts for 29\% of the total population\footnote{See \url{https://success.crowdflower.com/hc/en-us/articles/202703345-Crowd-Demographics}, the calculation was done on 19 December 2015.}.

\textbf{Accuracy} As described in the previous section, as a test question Workers were asked to copy and paste or type the name of the street as part of their submission in each task. Being the question this simple, error was not accepted the requested accuracy was 99\%\footnote{CrowdFlower does not allow the Requester to set target accuracy to 100\%.}.

\textbf{Judgements} In groups of 5 per road, repeated until consensus is reached, by different Workers without repetition. Each Worker is allowed to contribute to as many tasks as possible.

\textbf{Behaviour} Each Worker was paid for 1 task, and 1 task is made of 1 street to survey.

\textbf{Reward / Time Limits} The reward was 0.20 US Dollars per task. Workers requiring less than 90 seconds per task were considered at high risk of being malicious and excluded to perform additional tasks. CrowdFlower imposes a time limit of 30 minutes maximum per task.

\subsection{Implementation}

\begin{sloppypar} %% see http://tex.stackexchange.com/questions/54946/how-to-break-long-url-in-an-item#comment114992_54949
The code implementing WODA and detailed documentation can be found on GitHub, starting from the instructions at \url{https://github.com/Digital-Contraptions-Imaginarium/OLAF-yr2_lab/blob/gh-pages/README.md}.
\end{sloppypar}
\section{Evaluation}

\subsection{Data}

WODA was deployed for a specific geographic sample, that is the same five OS 1:10,000 raster tiles that were previously used in literature to analyse the performance of GI crowdsourcing and originally selected by Haklay in \cite{Haklay:2010vs} to assess OSM\footnote{The tiles are: TQ37ne, TQ28ne, TQ29nw, TQ26se and TQ36nw.}. This is an area of ~113 $ km^2 $ of Greater London that includes 3,982 named roads. 

\subsection{Evaluation metrics}

The evaluation is aimed at observing WODA's effectiveness at producing house numbers for the sample. Given the premises described in chapter \ref{introduction}, a financial metric is used to measure performance, that is the average cost per road of using crowdsourcing for data production. 

\subsection{Results}

\textbf{Ingestion of primary data sources} The data obtained by implementing the approach described in section \ref{crowdsourcing-olaf} successfully populated house numbers from LRPP for 82\% of the streets in OSON.

\textbf{House number inference} The conditions necessary to apply the inference algorithms, before using any crowdsourcing, were verified for 74\% of roads. Applied to these, algorithms \ref{algo:inference-numbers} and \ref{algo:inference-numbers-suffix} generated ~113k house numbers in addition to the already known ~111k (+102\%). 

\textbf{House number crowdsourcing} Stop condition {\it s.2} - with 118 repeat judgments vs 117 first judgements - was verified at the end of three crowdsourcing rounds.  The data for only 4 roads only was collected successfully, with an average cost of 9.00 USD per road. 

To better examine any trends in Worker behaviour and consensus through iterations, three additional rounds were run, too. 

More than 12\% of Workers were caught submitting judgements earlier than the 90 seconds limit, and were excluded from contributing further. 19\% failed the test of copying the name of the road in the form at least once, and thus were identified as not credible and their judgements ignored. The average time it took for credible Workers to complete the tasks was 6:14. 83\% of them were always faster than 3:00, that means that they never watched the instructions video in full before submitting.

The extra three rounds showed how agreement on most roads not only failed to converge, but stalled or worsened with new iterations.\footnote{See \url{https://github.com/Digital-Contraptions-Imaginarium/OLAF-yr2_lab/tree/gh-pages/docs#fleiss-kappa-vs-iterations}.} 


\section{Discussion}

From a merely technical perspective, the experience of implementing WODA confirmed that it is feasible to implement a system that supports non-trivial human-machine hybrid workflows while relying only on the scalability and availability of third party services only.

Conversely, from a functional point of view, many of the chosen services' characteristics - and of the crowdsourcing platform in particular - ended up not to be compatible with the needs of our design. So, while we achieved avoiding custom Worker-facing software components by relying exclusively on CrowdFlower and Google Map's native features, we were also forced to implement many ad hoc workarounds. Several of these arrangements were far from ideal and affected negatively the performance of the overall system, to a point compromising its effectiveness and advantages. 

For example, not having the possibility to rely on CrowdFlower's own metric (quorum vs Fleiss' kappa) did not allow us to calculate consensus as new judgements came into the system, but only offline, between crowdsourcing rounds. This forced us to collect more judgements than necessary. In turn, we also lost the option to use CrowdFlower's features that prevent Workers to judge the same item repeatedly, causing an overwhelming volume of unwanted judgements that stopped the experiment very early in respect to our expectations. 

An equally pressing issue is the choice of crowdsourcing task, that failed to catalyse the contributors' agreement and produce results that are statistically credible. This suggests a substantial doubt on the effectiveness of crowdsourcing survey activities of this sort. We reckon that the key cause was a combination of (i) conventional cheating, (ii) the complexity and open-ended nature of the surveying task and (iii) some degree of sloppiness of participants even when in good faith. When aiming at such a high consensus target, as in our case, even one single point of disagreement can substantially negatively impact the metric.

Cheating in crowdsourcing does not require explanation and is common to most systems. The way the task was designed, it was easy for participants to just state that the surveyed street offered no house numbers. 

The act of surveying a location - in person as in the interactive imagery - is not trivial. Some of its dynamics may seem intuitive, however not all participants would necessarily grasp them, even after watching the instructions video. Gottlieb {\it et al.} in \cite{Gottlieb:2012fh} faced similar difficulty, e.g. in exploring crowdsourcing to geolocate places in videos.

In terms of sloppiness of the surveys, even in good faith many Workers may have been content when finding house numbers that looked "small enough" or "high enough", and interrupted their search there, despite what was stated in the instructions. 

Finally, any discussion should still be filtered through a cost / benefit examination: what is a "reasonable cost" of producing one address with a target degree of confidence. The volume of roads we could enable inference for was too little to make such considerations.

\section{Conclusions}

We have presented WODA: a platform to integrate geospatial open data, computation and original human contribution to create original data, using human-machine hybrid workflows. To maximise the platform's scalability and availability, our design has relied as much as possible on the native features of mainstream SaaS services such as CrowdFlower and Google Maps. We have then implemented the platform to tackle components of one specific real life problem, that is the creation of OLAF: the list of all valid UK addresses. Where the conditions to enable computation were not verified, we deployed the platform to use paid microtask crowdsourcing to create the missing input data.

The evaluation of the platform has showed how critical the crowdsourcing component of the system is, particularly when it needs to support contributors in performing more complex activities than what is common in microtasking, such as surveying interactive imagery of locations. 

Moreover, our experience with CrowdFlower has also showed how third party services' native features may be intrinsically inconsistent with the needs of a workflow, de facto forcing the system designer to write ad hoc software that uses their API instead. This adds complexity and points of failure to the overall solution that could be avoided instead, and possibly compromises scalability and availability. 

Our plans for future work include exploring alternative configurations of pre-existing open data, computation and human contribution where the crowdsourcing component can be designed to achieve a higher degree of success. We also intend to share our work with the community of geospatial practitioners in London at one of their upcoming regular meetings, to gather feedback and ideas for further development.

Another interesting direction of research is to both (a) investigate how to re-design WODA to make the best use of the crowdsourcing provider's native functionality, and (b) work with them to implement those missing features and/or work around the limitations we identified during this work.


\textbf{Acknowledgements.} [WRITE THIS, NOTHING REVEALING IF THE REVIEW IS BLIND]

\documentclass{article}
\usepackage{url}
\begin{document}
\nocite{*}
\bibliography{main}
\bibliographystyle{splncs03}
\end{document}


\end{document}

