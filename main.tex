\documentclass{llncs}
\usepackage[utf8]{inputenc}

%% Adds URL support
\usepackage{url}

\title{Human-machine hybrid workflow for the augmentation of open datasets}

\author{Gianfranco Cecconi}
\institute{University of Southampton \email{gc1a13@soton.ac.uk}}


\date{December 2015}

\begin{document}

\maketitle

\section{Introduction}

\subsection{Crowdsourcing}

Crowdsourcing is {[}...{]}. More generally {[}BLAH BLAH SOCIAL MACHINES AS POSSIBLY ONE OF THE ONLY WAYS TO SOLVE PROBLEMS LIKE THIS + SOME EXCUSE TO CITE \cite{OReilly:2015uo}{]}

\subsection{Crowdsourcing Geographic Information}

Among the many successful applications of crowdsourcing is the collection and maintenance of geospatial data, or "Geographic Information" (GI), as it is commonly referred to in literature.

Although it is commonly recognised for GI to have a significant economic and social value \cite{Sui:2012uf}[THIS IS A REFERENCE TO AN ENTIRE BOOK, LIKELY UNSUITABLE], the effort National Mapping and Cadastre Agencies (NMCAs) put into producing and updating cartography has been in decline for several decades \cite{ESTES:1994vz}. In the U.S., for example, the Geological Survey no longer attempts to update its maps on a regular basis and the National Research Council promotes a vision in which {[}...{]} \cite{Committee:1993vp}.

{[}Somewhere in the literature someone said that the emergence of VGI was actually suggested by one of those authorities{]}

Crowdsourcing GI is then seen as the cost effective and good enough*** solution to this problem. The phenomenon of {\it Volunteered} Geographic Information (VGI) in particular was studied extensively since the term was coined \cite{Goodchild:2007vt}. Developing understanding of VGI was made possible by the success of services such as Wikimapia\footnote{\url{http://wikimapia.org/}.} or OpenStreetMap\footnote{\url{http://www.openstreetmap.org/}.}. The latter is likely the best known VGI-based mapping service available today.

\subsection{Open data}

Open data is data that anyone can access, use and share\footnote{\url{http://theodi.org/faq}.}. 

NMCAs, as governments and private organisations, are becoming more sensitive to the opportunities arising from publishing and re-using open data. In Great Britain, for example, since 2015 the local NMCA Ordnance Survey\footnote{\url{http://www.ordnancesurvey.co.uk/}.} has released in the open a substantial volume of data that was previously available to the public as commercial products only, e.g. "Open Names"\footnote{\url{https://www.ordnancesurvey.co.uk/business-and-government/products/os-open-names.html}.}, a place name index, and "Open Roads"\footnote{\url{https://www.ordnancesurvey.co.uk/business-and-government/products/os-open-roads.html}.}, the generalised geometry and network connectivity of the road network.

The availability of such high quality and authoritative sources becomes a substantial catalyser for the creation of new GI. There were geospatial data could only be created from scratch - as in the case of OpenStreetMap when it was started in the UK in 2007 - it is now possible to rather focus the effort of the crowd on complementing the GI that is already available.

\subsection{Challenges}

Challenges remain in making the most out of the advantage described above, that is a prerogative of GI in those countries where geospatial open data is published. In general, it is useful and relevant to identify what characteristics of crowdsourcing systems assure better performance in those situation where pre-existing data is available for integration, whatever the data represents. 

In this paper we propose methods for improving the performance of such crowdsourcing systems, measured as a combination of speed of production and quality of output. We leverage three factors in particular that were the subject to our experiments: a) the crowd capability to choose their tasks, b) the motivation arising from visibility of collective effort and c) the dynamic allocation of monetary rewards.

{[}
Add
\begin{itemize}
	\item {[}rationale of why we thought this was relevant, some justification in literature review{]}
	\item {[}novelty of what we propose{]}
\end{itemize}
{]}

{(}...{)}

{[}Definition of address{]}

Many complementary and/or alternative strategies are possible and need to be used jointly to build OLAF. Among these is the opportunity to re-use published open datasets of addresses so to infer the existence of more addresses. 

E.g. it is intuitive that if some source refers to the existence of house numbers 3, 5 and 9 in some street, and all are associated to the same postcode, it is very likely that number 7 exists as well and is associated to the same postcode\footnote{House numbering and postcode association are heavily dependant of the conventions used in the country the problem is applied to. This paper always refers to the UK conventions.}. It is shown experimentally that the method is effective, as it can produce large volumes of addresses from available open data\footnote{This was tested against the single largest known source of addresses open data for England and Wales: Land Registry's "Price Paid Data". See \url{https://www.gov.uk/government/collections/price-paid-data}.}.

The experiments described in this paper implement the above strategy only, and  use crowdsourcing to validate sets of inferred addresses, as participants are asked to virtually survey the streets using pictures sourced from Google Street View.

It has to be noted that not all existing house numbers are visible by surveying a street. E.g. there are no obligations in the UK to affix a house number or house name sign. Moreover, some of the house numbers may be associated to dwellings that are not visible unless the property is accessed, beyond what Google Street View's photos can capture.

{(}...{)}








\begin{thebibliography}{1}
\providecommand{\url}[1]{\texttt{#1}}
\providecommand{\urlprefix}{URL }

\bibitem{Committee:1993vp}
Committee, M.S., Mapping Science~Committee, N.R.C.: {Toward a coordinated
  spatial data infrastructure for the nation} (1993),
  \url{http://books.google.com/books?hl=en&lr=&id=t6iUtE5M-kQC&oi=fnd&pg=PA1&dq=Toward+a+Coordinated+Spatial+Data+Infrastructure+for+the+Nation&ots=eL8wjGdRaX&sig=L12qNHmOXpYgKIM8Fp3gPpUSDns}

\bibitem{ESTES:1994vz}
Estes, J.E., Mooneyhan, D.W.: {Of Maps and Myths}. Photogrammetric Engineering
  and Remote Sensing  60(5),  517--524 (May 1994),
  \url{http://gateway.webofknowledge.com/gateway/Gateway.cgi?GWVersion=2&SrcAuth=mekentosj&SrcApp=Papers&DestLinkType=FullRecord&DestApp=WOS&KeyUT=A1994NW25600004}

\bibitem{Goodchild:2007vt}
Goodchild, M.F.: {Citizens as sensors: the world of volunteered geography}.
  GeoJournal  (2007),
  \url{http://link.springer.com/article/10.1007/s10708-007-9111-y}

\bibitem{OReilly:2015uo}
O'Reilly, T.: {The WTF Economy {\textemdash} What{\textquoteright}s The Future
  of Work? {\textemdash} Medium} pp. 1--7 (Jul 2015),
  \url{https://medium.com/the-wtf-economy/the-wtf-economy-a3bd5f52ef00}

\bibitem{Sui:2012uf}
Sui, D., Elwood, S., Goodchild, M.: {Crowdsourcing geographic knowledge:
  volunteered geographic information (VGI) in theory and practice} (2012),
  \url{http://books.google.com/books?hl=en&lr=&id=SSbHUpSk2MsC&oi=fnd&pg=PR7&dq=Crowdsourcing+Geographic+Knowledge&ots=J4Ux8AyBa3&sig=HJTMrc9O58rx2dHgBasDRGVcs8s}

\end{thebibliography}


\end{document}

