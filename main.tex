\documentclass{llncs}
\usepackage[utf8]{inputenc}

%% Adds URL support
\usepackage{url}

\title{Human-machine hybrid workflow for the augmentation of open datasets}

\author{Gianfranco Cecconi}
\institute{University of Southampton \email{gc1a13@soton.ac.uk}}


\date{December 2015}

\begin{document}

\maketitle

\section{Introduction}

\subsection{Crowdsourcing}

Crowdsourcing is {[}...{]}. More generally {[}BLAH BLAH SOCIAL MACHINES AS POSSIBLY ONE OF THE ONLY WAYS TO SOLVE PROBLEMS LIKE THIS + SOME EXCUSE TO CITE \cite{OReilly:2015uo}{]}

\subsection{Crowdsourcing Geographic Information}

Among the many successful applications of crowdsourcing is the collection and maintenance of geospatial data, or "Geographic Information" (GI), as it is commonly referred to in literature.

Although it is commonly recognised for GI to have a significant economic and social value \cite{Sui:2012uf}[THIS IS A REFERENCE TO AN ENTIRE BOOK, LIKELY UNSUITABLE], the effort National Mapping and Cadastre Agencies (NMCAs) put into producing and updating cartography has been in decline for several decades \cite{ESTES:1994vz}. In the U.S., for example, the Geological Survey no longer attempts to update its maps on a regular basis and the National Research Council promotes a vision in which {[}...{]} \cite{Committee:1993vp}.

{[}Somewhere in the literature someone said that the emergence of VGI was actually suggested by one of those authorities{]}

Crowdsourcing GI is then seen as the cost effective and good enough*** solution to this problem. The phenomenon of {\it Volunteered} Geographic Information (VGI) in particular was studied extensively since the term was coined \cite{Goodchild:2007vt}. Developing understanding of VGI was made possible by the success of services such as Wikimapia\footnote{\url{http://wikimapia.org/}.} or OpenStreetMap\footnote{\url{http://www.openstreetmap.org/}.}. The latter is likely the best known VGI-based mapping service available today.

\subsection{Open data}

Open data is data that anyone can access, use and share\footnote{\url{http://theodi.org/faq}.}. 

NMCAs, as governments and private organisations, are becoming more sensitive to the opportunities arising from publishing and re-using open data. In Great Britain, for example, since 2015 the local NMCA Ordnance Survey\footnote{\url{http://www.ordnancesurvey.co.uk/}.} has released in the open a substantial volume of data that was previously available to the public as commercial products only, e.g. "Open Names"\footnote{\url{https://www.ordnancesurvey.co.uk/business-and-government/products/os-open-names.html}.}, a place name index, and "Open Roads"\footnote{\url{https://www.ordnancesurvey.co.uk/business-and-government/products/os-open-roads.html}.}, the generalised geometry and network connectivity of the road network.

The availability of such high quality and authoritative sources becomes a substantial catalyser for the creation of new GI. There were geospatial data could only be created from scratch - as in the case of OpenStreetMap when it was started in the UK in 2007 - it is now possible to rather focus the effort of the crowd on complementing the GI that is already available.

\begin{thebibliography}{1}
\providecommand{\url}[1]{\texttt{#1}}
\providecommand{\urlprefix}{URL }

\bibitem{Committee:1993vp}
Committee, M.S., Mapping Science~Committee, N.R.C.: {Toward a coordinated
  spatial data infrastructure for the nation} (1993),
  \url{http://books.google.com/books?hl=en&lr=&id=t6iUtE5M-kQC&oi=fnd&pg=PA1&dq=Toward+a+Coordinated+Spatial+Data+Infrastructure+for+the+Nation&ots=eL8wjGdRaX&sig=L12qNHmOXpYgKIM8Fp3gPpUSDns}

\bibitem{ESTES:1994vz}
Estes, J.E., Mooneyhan, D.W.: {Of Maps and Myths}. Photogrammetric Engineering
  and Remote Sensing  60(5),  517--524 (May 1994),
  \url{http://gateway.webofknowledge.com/gateway/Gateway.cgi?GWVersion=2&SrcAuth=mekentosj&SrcApp=Papers&DestLinkType=FullRecord&DestApp=WOS&KeyUT=A1994NW25600004}

\bibitem{Goodchild:2007vt}
Goodchild, M.F.: {Citizens as sensors: the world of volunteered geography}.
  GeoJournal  (2007),
  \url{http://link.springer.com/article/10.1007/s10708-007-9111-y}

\bibitem{OReilly:2015uo}
O'Reilly, T.: {The WTF Economy {\textemdash} What{\textquoteright}s The Future
  of Work? {\textemdash} Medium} pp. 1--7 (Jul 2015),
  \url{https://medium.com/the-wtf-economy/the-wtf-economy-a3bd5f52ef00}

\bibitem{Sui:2012uf}
Sui, D., Elwood, S., Goodchild, M.: {Crowdsourcing geographic knowledge:
  volunteered geographic information (VGI) in theory and practice} (2012),
  \url{http://books.google.com/books?hl=en&lr=&id=SSbHUpSk2MsC&oi=fnd&pg=PR7&dq=Crowdsourcing+Geographic+Knowledge&ots=J4Ux8AyBa3&sig=HJTMrc9O58rx2dHgBasDRGVcs8s}

\end{thebibliography}


\end{document}

