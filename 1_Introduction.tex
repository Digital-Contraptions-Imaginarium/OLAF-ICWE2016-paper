\section{Introduction}

Geospatial data is a critical component of many applications. Its value is universally recognised as instrumental to economic development, yet little is readily available and subject to high costs or restrictive licensing. 

The availability of geospatial data under open licensing for the public good is under threat. Its production and maintenance is expensive and National Mapping and Cadastre Agencies (NMCAs) worldwide - the traditional suppliers of such data - are cutting investment into producing and updating cartography and other assets \cite{ESTES:1994vz}. In the U.S., for example, the Geological Survey (USGS) no longer updates maps on a regular basis. 

A vision is emerging in which geospatial data is gathered from a pool of heterogeneous open sources including government and private sector, and consolidated into new data products and services as needed. In the US, the National Research Council promotes such a vision\footnote{M. S. Committee and N. R. C. Mapping Science Committee, Toward a coordinated spatial data infrastructure for the nation. 1993.}.

This new paradigm requires process and technological transformation. Among the tools available to whoever takes responsibility for such consolidation is hybrid human-machine workflows that cost-effectively integrate pre-existing open data, computation and original contributions by human participants - typically through crowdsourcing. 

As a useful tool for geospatial researchers and practitioners involved in this transformation, in this paper we present "WODA" (Workflows for Open Data Augmentation): a platform implementing hybrid human-machine workflows that can be configured to produce different kinds of data, support different types of crowdsourcing models (tasks, participation, rewards to participants etc.) and is made available as open source. 

We then apply and evaluate WODA to deal with a specific use case: the creation of the list of all valid addresses in the UK, or the "Open Legal Address File", which has received a lot of attention in the country as an example of simple, though critical, geographic data that is available only as a commercial product.
