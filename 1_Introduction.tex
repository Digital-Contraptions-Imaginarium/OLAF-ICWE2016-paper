\section{Introduction}

Geospatial data is a critical component of many applications. Its value is universally recognised to be instrumental to economic development, yet not all of it is readily available, and, when it is, its use is associated with high costs or restrictive licensing. 

The availability of geospatial data under an open licensing in particular is not only limited, but the effort national mapping and cadastre agencies (NMCAs) worldwide put into producing and updating cartography and their other assets for the public good has been in decline for several decades \cite{ESTES:1994vz}. In the U.S., for example, the Geological Survey (USGS) no longer attempts to update its maps on a regular basis and the National Research Council promotes a vision in which the creation and curation geospatial data is no longer centralised but rather shared and consolidated from many governmental and private sector sources \cite{Committee:1993vp}.

In this document, we look at one possible solution to these problems and propose a platform to create and curate geospatial data that integrates pre-existing open data, computation and original contributions by human participants - typically through crowdsourcing - in hybrid human-machine workflows. The platform can be configured to produce different kinds of data, support different types of crowdsourcing models (tasks, participation, rewards to participants etc) and is made available as open source.

We then focus on a specific use case: the creation of the list of all valid addresses in the UK, or the "Open Legal Address File" ("OLAF"), which has received a lot of attention in the country as it a great example of simple, though critical, geographic data that is unfortunately available only as a commercial product. We use the platform to implement one of the possible workflows to create parts of OLAF, and evaluate its performance. 

We believe this could be a useful tool for geospatial researchers and practitioners interested in systematically using crowdsourcing in their daily work.
