\section{Introduction}

[ELENA'S SUGGESTED FEEL FOR THIS SECTION IS:] geospatial data is a critical component of many applications. Yet not all of it is readily available, and when available, its use is associated with high costs. We look at crowdsourcing, in particular microtask crowdsourcing, as a means to offer geospatial data for the public good. We focus on a specific use case, the OLAF, which has received a lot of attention in the UK and propose a platform to outsource OLAF microtasks to a crowd. The platform can be configured to support different types of tasks and participation models (paid, unpaid, individual vs team etc) and is available as open source. We evaluate it on an experiment ... We believe this could be a useful tool for geospatial researchers and practitioners interested in systematically using crowdsourcing in their daily work.

\subsection{Crowdsourcing}

    Crowdsourcing is {[}...{]}. More generally {[}SOME LIGHTWEIGHT REFERENCE TO WHAT A SOCIAL MACHINE IS{]} [{[}BLAH BLAH SOCIAL MACHINES AS POSSIBLY ONE OF THE ONLY WAYS TO SOLVE PROBLEMS LIKE THIS + SOME EXCUSE TO CITE \cite{OReilly:2015uo}{]}

