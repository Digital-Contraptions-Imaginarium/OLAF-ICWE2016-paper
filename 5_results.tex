\section{Results}

The result of the experiment is summarised in tables \ref{table:distribution-of-workers} and \ref{table:judgements-summary} below. 

After three consecutive iterations and a total of 150 judgements per road by 80 Workers, the number of total duplicate judgements by reliable Workers was higher than the number of unique judgements, hence triggering the stop condition. 

18.75\% of Workers failed the simple test of copying the name of the road in the form, and were not considered trustworthy. 

Agreement could be reached on four streets only at the end of those three rounds. All roads were identified as not having house numbers.

Collection was reprised at a later point in time to attempt leveraging a different Worker base, through three additional rounds. The judgements of 41 additional Workers were collected, achieving consensus on three more roads. 

\begin{table}[]
\centering
\begin{tabular}{|l|c|c|l|l|}
\hline
                   & \multicolumn{1}{l|}{\begin{tabular}[c]{@{}l@{}}Tot. no. of \\ Workers\end{tabular}} & \multicolumn{1}{l|}{\begin{tabular}[c]{@{}l@{}}No. of new \\ Workers\end{tabular}} & \begin{tabular}[c]{@{}l@{}}No. of \\ reliable\\ Workers\end{tabular} & \begin{tabular}[c]{@{}l@{}}\% of \\ reliable \\ Workers\end{tabular} \\ \hline
First three rounds & 80                                                                                  & -                                                                                  & 65                                                                   & 81.25\%                                                              \\ \hline
All six rounds     & 121                                                                                 & 41                                                                                 & 97                                                                   & 80.16\%                                                              \\ \hline
\end{tabular}
\caption{Distribution of Workers across rounds}
\label{table:distribution-of-workers}
\end{table}



\begin{table}[]
\centering
\begin{tabular}{|l|l|l|l|}
\hline
                   & \begin{tabular}[c]{@{}l@{}}No. of \\ reliable\\ Workers\end{tabular} & \begin{tabular}[c]{@{}l@{}}No. of non-duplicate\\ judgements\end{tabular} & \begin{tabular}[c]{@{}l@{}}No. of roads\\ where consensus\\ was achieved\end{tabular} \\ \hline
First three rounds & 65                                                                   & 117                                                                       & 4                                                                                     \\ \hline
All six rounds     & 97                                                                   & 227                                                                       & 7                                                                                     \\ \hline
\end{tabular}
\caption{Judgement numbers and consensus summary}
\label{table:judgements-summary}
\end{table}

The graph below illustrates how Fleiss kappa variated...


