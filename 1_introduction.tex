\section{Introduction}

\subsection{Crowdsourcing}

    Crowdsourcing is {[}...{]}. More generally {[}SOME LIGHTWEIGHT REFERENCE TO WHAT A SOCIAL MACHINE IS{]} [{[}BLAH BLAH SOCIAL MACHINES AS POSSIBLY ONE OF THE ONLY WAYS TO SOLVE PROBLEMS LIKE THIS + SOME EXCUSE TO CITE \cite{OReilly:2015uo}{]}

\subsection{Crowdsourcing Geographic Information}

    Among the many applications of crowdsourcing is the collection and maintenance of geospatial data, or "Geographic Information" (GI).
    
    Although it is commonly recognised for GI to have a significant economic and social value \cite{Sui:2012uf}[THIS IS A REFERENCE TO AN ENTIRE BOOK, LIKELY UNSUITABLE], the effort national mapping and cadastre agencies (NMCAs) worldwide put into producing and updating cartography has been in decline for several decades \cite{ESTES:1994vz}. In the U.S., for example, the Geological Survey (USGS) no longer attempts to update its maps on a regular basis and the National Research Council promotes a vision in which {[}...{]} \cite{Committee:1993vp}.
    
    {[}Somewhere in the literature someone said that the emergence of VGI was actually suggested by one of those authorities{]}
    
    Crowdsourcing GI is then seen as the cost effective and "good enough" solution to this problem [CITATION OF SOME SCHOLAR SAYING THAT GOOD ENOUGH IS... GOOD ENOUGH]. The phenomenon of {\it Volunteered} Geographic Information (VGI) in particular was studied extensively since the term was coined \cite{Goodchild:2007vt}. Developing understanding of VGI was made possible by the success of services such as Wikimapia\footnote{\url{http://wikimapia.org/}.} or OpenStreetMap\footnote{\url{http://www.openstreetmap.org/}.}. The latter is likely the best known VGI-based mapping service available today.

\subsection{Open data and open Geographic Information}

    Open data is data that anyone can access, use and share\footnote{\url{http://theodi.org/faq}.}. 
    
    NMCAs, as governments and private organisations, are becoming more sensitive to the opportunities arising from publishing and re-using open data. In Great Britain, for example, since 2015 the local NMCA Ordnance Survey\footnote{\url{http://www.ordnancesurvey.co.uk/}.} has released in the open a substantial volume of data that was previously available to the public as commercial products only, e.g. "Open Names"\footnote{\url{https://www.ordnancesurvey.co.uk/business-and-government/products/os-open-names.html}.}, a place-name index, and "Open Roads"\footnote{\url{https://www.ordnancesurvey.co.uk/business-and-government/products/os-open-roads.html}.}, the generalised geometry and network connectivity of the road network.
    
    The availability of such high quality and authoritative sources becomes a substantial enabler for the creation of new, original geospatial data. There where GI could only be created from scratch - as in OpenStreetMap's case when it was started in the UK in 2007 - it is now possible to rather focus the effort of the crowd on complementing what is already available, or augmenting it.

\subsection{Crowdsourcing open data}

    The work described in this paper took place in the context of a larger research programme aimed at assessing the feasibility of building original open data by using non-expert human contributions, technology systems and, where available, other pre-existing open data. 
    
    Although most of the effort of producing and releasing open data is commonly expected of governments and businesses, there are industry sectors and domains of knowledge where resistance to change can halt or substantially slow down this process. Demand and offer for open data is strongly suppressed, for example, due to failure in recognising open data-enabled business models, restrictive legislative and patent systems, or charging for public datasets \cite{shadboltpaf}. 
    
    The assumption at the base of our research is that the people's contribution is necessary to address this problem. People can be enabled, through technology, to capture and curate data, alongside what is already published by governments and businesses. Crowdsourcing is just one of the formulae by which socio-technical systems can be built for this purpose. This contribution is instrumental to delivering and curating the open data needed to build and operate a comprehensive {\it national information infrastructure} (NII).

\subsection{Challenges of crowdsourcing geospatial data}

    Two groups of challenges are relevant to the availability of good quality geospatial open data: on one side, exploiting it to improve the quality and enhance the functionality of pre-existing VGI initiatives, and, on the other, unleash completely new products and services.
    
    [CHALLENGE OF MAINTENANCE, AFTER COLLECTION]
    
    [SHOULD I WRITE MORE ABOUT THE OTHER CHALLENGES, TOO?]
    
    Completeness is a key element in the quality of geospatial data. In this paper we propose a method to improve the coverage of existing geospatial data where - for any reason - it is not possible to get volunteers to survey one specific geographical area or provide more detail than what is available already. This was described, for example, by **** , observing how OpenStreetMap volunteers naturally tend to avoid ***.
    
    {[}
    Add
    \begin{itemize}
    	\item {[}rationale of why we thought this was relevant, some justification in literature review{]}
    	\item {[}novelty of what we propose{]}
    \end{itemize}
    {]}
    
    {(}...{)}
