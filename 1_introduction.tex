\section{Introduction}

\subsection{Crowdsourcing}

    Crowdsourcing is {[}...{]}. More generally {[}SOME LIGHTWEIGHT REFERENCE TO WHAT A SOCIAL MACHINE IS{]} [{[}BLAH BLAH SOCIAL MACHINES AS POSSIBLY ONE OF THE ONLY WAYS TO SOLVE PROBLEMS LIKE THIS + SOME EXCUSE TO CITE \cite{OReilly:2015uo}{]}

\subsection{Crowdsourcing Geographic Information}

    Among the many applications of crowdsourcing is the collection and maintenance of geospatial data, or "Geographic Information" (GI), as it is commonly referred to in literature.
    
    Although it is commonly recognised for GI to have a significant economic and social value \cite{Sui:2012uf}[THIS IS A REFERENCE TO AN ENTIRE BOOK, LIKELY UNSUITABLE], the effort national mapping and cadastre agencies (NMCAs) worldwide put into producing and updating cartography has been in decline for several decades \cite{ESTES:1994vz}. In the U.S., for example, the Geological Survey (USGS) no longer attempts to update its maps on a regular basis and the National Research Council promotes a vision in which {[}...{]} \cite{Committee:1993vp}.
    
    {[}Somewhere in the literature someone said that the emergence of VGI was actually suggested by one of those authorities{]}
    
    Crowdsourcing GI is then seen as the cost effective and "good enough" solution to this problem [CITATION OF SOME SCHOLAR SAYING THAT GOOD ENOUGH IS... GOOD ENOUGH]. The phenomenon of {\it Volunteered} Geographic Information (VGI) in particular was studied extensively since the term was coined \cite{Goodchild:2007vt}. Developing understanding of VGI was made possible by the success of services such as Wikimapia\footnote{\url{http://wikimapia.org/}.} or OpenStreetMap\footnote{\url{http://www.openstreetmap.org/}.}. The latter is likely the best known VGI-based mapping service available today.

\subsection{Open data}

    Open data is data that anyone can access, use and share\footnote{\url{http://theodi.org/faq}.}. 
    
    NMCAs, as governments and private organisations, are becoming more sensitive to the opportunities arising from publishing and re-using open data. In Great Britain, for example, since 2015 the local NMCA Ordnance Survey\footnote{\url{http://www.ordnancesurvey.co.uk/}.} has released in the open a substantial volume of data that was previously available to the public as commercial products only, e.g. "Open Names"\footnote{\url{https://www.ordnancesurvey.co.uk/business-and-government/products/os-open-names.html}.}, a place-name index, and "Open Roads"\footnote{\url{https://www.ordnancesurvey.co.uk/business-and-government/products/os-open-roads.html}.}, the generalised geometry and network connectivity of the road network.
    
    The availability of such high quality and authoritative sources becomes a substantial enabler for the creation of new GI. There were geospatial data could only be created from scratch - as in OpenStreetMap's case when it was started in the UK in 2007 - it is now possible to rather focus the effort of the crowd on complementing the GI that is already available.

\subsection{Crowdsourcing open data}

    This work took place in the context of a larger research programme aimed at assessing the feasibility of building original open data by using non-expert human contributions, technology systems and, where available, other pre-existing open data. 
    
    Although most of the effort of producing and releasing open data is commonly expected of governments and businesses, there are industry sectors and domains of knowledge where resistance to change can halt or substantially slow down this process. Demand and offer for open data is strongly suppressed, for example, due to failure in recognising open data-enabled business models, restrictive legislative and patent systems, or charging for public datasets \cite{shadboltpaf}. 
    
    The assumption at the base of our research is that the people's contribution is necessary to address this problem. People can be enabled, through technology, to capture and curate data, alongside what is already published by governments and businesses. Crowdsourcing is just one of the formulae by which socio-technical systems can be built for this purpose. This contribution is instrumental to delivering and curating the open data needed to build and operate a comprehensive {\it national information infrastructure} (NII).

\subsection{Challenges of crowdsourcing geospatial data}

    Two groups of challenges are relevant to the availability of good quality geospatial open data: on one side, exploiting it to improve the quality and enhance the functionality of pre-existing VGI initiatives, and, on the other, unleash completely new products and services.
    
    [CHALLENGE OF MAINTENANCE, AFTER COLLECTION]
    
    [SHOULD I WRITE MORE ABOUT THE OTHER CHALLENGES, TOO?]
    
    Completeness is a key element in the quality of geospatial data. In this paper we propose a method to improve the coverage of existing geospatial data where - for any reason - it is not possible to get volunteers to survey one specific geographical area or provide more detail than what is available already. This was described, for example, by **** , observing how OpenStreetMap volunteers naturally tend to avoid ***.
    
    {[}
    Add
    \begin{itemize}
    	\item {[}rationale of why we thought this was relevant, some justification in literature review{]}
    	\item {[}novelty of what we propose{]}
    \end{itemize}
    {]}
    
    {(}...{)}

\subsection{The Open Legal Address File}

    The main use case for this research is the attempt of creating a new geospatial dataset that is functionally equivalent to the "Postcode Address File" or "PAF"\footnote{PAF is a registered trademark by Royal Mail plc. For convenience we won't show the registered trademark sign "\textregistered" in this document every time we refer to it.}: a commercial dataset listing all known valid addresses and postcodes for the UK at a given point in time. 
    
    An act of law\footnote{The Postal Service Act 2000, part VII, article 116 \cite{postalserviceact2000}.} makes PAF ownership of the Royal Mail: the ex-postal service monopolist in the UK, now a public limited company with only a 30\% of shares controlled by the Government\footnote{The Postal Services Act 2011 plans to reduce the Government control down to 10\% \cite{postalserviceact2011}.}. The same act of law requires Royal Mail to make PAF available on "reasonable terms" to "any person who wishes to use it".
    
    To this day, though, the "reasonable terms" translated only into PAF being made available by Royal Mail and its resellers as a commercial product\footnote{As an end user who wants to access PAF's data one may incur Royal Mail licence fees going from \pounds0.012 per transaction on a public website to \pounds90,000 per year for unlimited internal use by a corporate group. Even charitable organisation may not take advantage of free PAF access unless their income is less than \pounds10m/year.}. This is considered an anomaly by many, as "reasonable terms for the external use of PAF data by third parties should be no more than the marginal cost of distribution (...)" \cite{odugresponse}. 

    To further limit the opportunities for PAF to be made open, in October 2013 Royal Mail and its assets were privatised, including PAF\footnote{\url{http://www.theguardian.com/uk-news/2014/mar/17/royal-mail-privatisation-ministers-rebuked-selling-data}}. The UK Parliament House of Commons' Public Administration Select Committee, just a few month later, called this "unacceptable and unnecessary" and recognised that PAF's "disposal for a short-term gain will impede economic innovation and growth" \cite{pascod}.

    This makes the opportunity of creating an alternative to PAF an ideal case study. We will call this the "Open Legal Address File", or "OLAF". The term "legal addresses" refers to all addresses that are, by law, in the public domain, hence have no restrictions in terms of intellectual property or privacy protection and can be published as open data\footnote{Legal addresses belong to either or both of the following two categories: a) they are the addresses of current or past UK residents who are or were registered on the public electoral register, and b) they are the addresses of past or present UK companies that are or were registered at the relevant British registrar, such as Companies House for England and Wales.}.

    {[}Definition of address{]}
    
    Many complementary and/or alternative strategies are possible and need to be used jointly to build OLAF. Among these is the opportunity to re-use published open datasets of addresses so to infer the existence of more addresses. 
    
    E.g. it is intuitive that if some source refers to the existence of house numbers 3, 5 and 9 in some street, and all are associated to the same postcode, it is very likely that number 7 exists as well and is associated to the same postcode\footnote{House numbering and postcode association are heavily dependant of the conventions used in the country the problem is applied to. This paper always refers to the UK conventions.}. It is shown experimentally that the method is effective, as it can produce large volumes of addresses from available open data\footnote{This was tested against the single largest known source of addresses open data for England and Wales: Land Registry's "Price Paid Data". See \url{https://www.gov.uk/government/collections/price-paid-data}.}.
    
    The experiments described in this paper implement the above strategy only, and  use crowdsourcing to validate sets of inferred addresses, as participants are asked to virtually survey the streets using pictures sourced from Google Street View.
    
    It has to be noted that not all existing house numbers are visible by surveying a street. E.g. there are no obligations in the UK to affix a house number or house name sign. Moreover, some of the house numbers may be associated to dwellings that are not visible unless the property is accessed, beyond what Google Street View's photos can capture.
    
    {(}...{)}