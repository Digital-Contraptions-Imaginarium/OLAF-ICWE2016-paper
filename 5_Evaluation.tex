\section{Evaluation}

The workflow was implemented on the platform in order to address the problem for a specific geographic sample, that is the same five OS 1:10,000 raster tiles that were randomly selected by Haklay in \cite{Haklay:2010vs} to assess OSM\footnote{The tiles are: TQ37ne, TQ28ne, TQ29nw, TQ26se and TQ36nw.}.

Figure \ref{fig:scope} shows the sample in the context of the overall OLAF problem. The sample is small enough in size to allow, if necessary, manual visual verification of the same streets the crowdworkers worked on, but anyway covers a large area of ~113 $ km^2 $ of Greater London, including 3,982 roads\footnote{The roads that were considered in scope are the ones that are classified by OSON as "named roads" and are fully included within any of the five tiles.}. Because of its long history, London offers a worst case scenario for evaluation, as the shape of the streets is not trivial, the streets have been subject to a lot of development over time etc. 

\subsection{House number inference}

The implementation of the approach described in section \ref{crowdsourcing-olaf} showed that LRPP offers house numbers for 82\% of the streets in scope. The algorithms described in \ref{inference-algorithms} can be applied to the 74\% of roads, generating ~113k house numbers. Figure \ref{fig:results-inference} shows [SOMETHING].

\begin{figure}[!ht]
    \begin{floatrow}
        \ffigbox{\includegraphics[width=0.505\textwidth]{scope.png}}{\caption{Scope for the evaluation.}\label{fig:scope}}
        \ffigbox{\includegraphics[width=0.475\textwidth]{inference-results.png}}{\caption{Results of inference application.}\label{fig:results-inference}}
   \end{floatrow}
\end{figure}

\subsection{House number crowdsourcing}

The result of the crowdsourcing experiments is summarised in tables \ref{table:distribution-of-workers} and \ref{table:judgements-summary} below. 

After three consecutive iterations and a total of 150 judgements per road by 80 Workers, the number of total duplicate judgements by reliable Workers was higher than the number of unique judgements, hence triggering the stop condition. 

18.75\% of Workers failed the simple test of copying the name of the road in the form, and were not considered trustworthy. 

Agreement could be reached on four streets only at the end of those three rounds. All roads were identified as not having house numbers.

Collection was reprised at a later point in time to attempt leveraging a different Worker base, through three additional rounds. The judgements of 41 additional Workers were collected, achieving consensus on three more roads. 

\begin{table}[]
\centering
\begin{tabular}{|l|c|c|l|l|}
\hline
                   & \multicolumn{1}{l|}{\begin{tabular}[c]{@{}l@{}}Tot. no. of \\ Workers\end{tabular}} & \multicolumn{1}{l|}{\begin{tabular}[c]{@{}l@{}}No. of new \\ Workers\end{tabular}} & \begin{tabular}[c]{@{}l@{}}No. of \\ reliable\\ Workers\end{tabular} & \begin{tabular}[c]{@{}l@{}}\% of \\ reliable \\ Workers\end{tabular} \\ \hline
First three rounds & 80                                                                                  & -                                                                                  & 65                                                                   & 81.25\%                                                              \\ \hline
All six rounds     & 121                                                                                 & 41                                                                                 & 97                                                                   & 80.16\%                                                              \\ \hline
\end{tabular}
\caption{Distribution of Workers across rounds}
\label{table:distribution-of-workers}
\end{table}



\begin{table}[]
\centering
\begin{tabular}{|l|l|l|l|}
\hline
                   & \begin{tabular}[c]{@{}l@{}}No. of \\ reliable\\ Workers\end{tabular} & \begin{tabular}[c]{@{}l@{}}No. of non-duplicate\\ judgements\end{tabular} & \begin{tabular}[c]{@{}l@{}}No. of roads\\ where consensus\\ was achieved\end{tabular} \\ \hline
First three rounds & 65                                                                   & 117                                                                       & 4                                                                                     \\ \hline
All six rounds     & 97                                                                   & 227                                                                       & 7                                                                                     \\ \hline
\end{tabular}
\caption{Judgement numbers and consensus summary}
\label{table:judgements-summary}
\end{table}

The graph below illustrates how Fleiss kappa variated...

\begin{figure}[!ht]
    \begin{floatrow}
        \ffigbox{\includegraphics[width=0.49\textwidth]{results-lowest-not-found.png}}{\caption{Roads for which the house numbers were not found. Submissions for the lowest house number.}\label{fig:results-lowest-not-found}}
        \ffigbox{\includegraphics[width=0.49\textwidth]{results-highest-not-found.png}}{\caption{Roads for which the house numbers were not found. Submissions for the highest house number.}\label{fig:results-highest-not-found}}
   \end{floatrow}
\end{figure}
        
\begin{figure}[!ht]
    \begin{floatrow}
        \ffigbox{\includegraphics[width=0.49\textwidth]{results-lowest-found.png}}{\caption{Roads for which the house numbers were found. Submissions for the lowest house number.}\label{fig:results-lowest-found}}
        \ffigbox{\includegraphics[width=0.49\textwidth]{results-highest-found.png}}{\caption{Roads for which the house numbers were found. Submissions for the highest house number.}\label{fig:results-highest-found}}
   \end{floatrow}
\end{figure}
        